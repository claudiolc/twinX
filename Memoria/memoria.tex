%%%%%%%%%%%%%%%%%%%%%%%%%%
% A template for Technical Report
% Prof. Siraj Shaikh
% Dr. Hoang Nga Nguyen
% Coventry University
% 2020
%%%%%%%%%%%%%%%%%%%%%%%%%%

%PLANTILLA
%\documentclass[oneside,openright,titlepage,numbers=noenddot,openany,headinclude,footinclude=true,
%cleardoublepage=empty,abstractoff,BCOR=5mm,paper=a4,fontsize=12pt,main=spanish]{scrreprt}

\documentclass[12pt]{article}
\usepackage[margin=1.2in]{geometry}
\usepackage[titletoc]{appendix}
\usepackage{graphicx}
\usepackage[space]{grffile}
\usepackage[utf8]{inputenc}
\usepackage[square,numbers]{natbib}
\usepackage{lipsum}
\usepackage{caption}
\usepackage{pdfpages}
\usepackage[spanish]{babel}
\usepackage{listings}
\usepackage{hyperref}
\usepackage[nonumberlist]{glossaries}
\usepackage{array}
\usepackage{subcaption}

% -------------------------------------------------------------------
% Ajustes de idioma
% -------------------------------------------------------------------
\addto\captionsspanish{%
	\renewcommand\appendixname{Anexo}
	\renewcommand\appendixpagename{Anexos}
}

% -------------------------------------------------------------------
% Plantilla
% -------------------------------------------------------------------



% Usa el paquete minted para mostrar trozos de código.
% Pueden seleccionarse el lenguaje apropiado y el estilo del código.
%\usepackage{minted}
%\usemintedstyle{colorful}
%\setminted{fontsize=\small}
%\setminted[haskell]{linenos=false,fontsize=\small}
%\renewcommand{\theFancyVerbLine}{\sffamily\textcolor[rgb]{0.5,0.5,1.0}{\oldstylenums{\arabic{FancyVerbLine}}}}

% Plantilla classicthesis
%\usepackage[beramono,eulerchapternumbers,linedheaders,parts,a5paper,dottedtoc,
%manychapters,pdfspacing]{classicthesis}

% Geometría y espaciado de párrafos.
%\setcounter{secnumdepth}{0}
%\usepackage{enumitem}
%\setitemize{noitemsep,topsep=0pt,parsep=0pt,partopsep=0pt}
%\setlist[enumerate]{topsep=0pt,itemsep=-1ex,partopsep=1ex,parsep=1ex}
%\usepackage[top=1in, bottom=1.5in, left=1in, right=1in]{geometry}
%\setlength\itemsep{0em}
%\setlength{\parindent}{0pt}
%\usepackage{parskip}

% Profundidad de la tabla de contenidos.
%\setcounter{secnumdepth}{3}
%
%% Archivos de configuración.
%\input{classicthesis-config} % En classicthesis-config.tex se almacenan las opciones propias de la plantilla.
%
%% Datos de portada
%\usepackage{titling} % Facilita los datos de la portada
%\author{Claudio López Carrascosa} 
%\date{\today}
%\title{Título}
%
%% Portada
%\usepackage{datetime}
\renewcommand\maketitle{
  \begin{titlepage}
    \begin{addmargin}[-2.5cm]{-3cm}
      \begin{center}
        \large  
        \hfill
        \vfill

        \begingroup
        \spacedallcaps{\thetitle}\\ \bigskip
        \includegraphics[width=5cm]{img/logo.png}
        \\ \bigskip
        \endgroup

        \spacedlowsmallcaps{\theauthor}

        \vfill

        Trabajo Fin de Grado \\ \medskip 
        Grado en Ingeniería Informática \\  \bigskip\bigskip


        \textbf{Tutora}\\
        Rosana Montes Soldado \\ \bigskip

        \spacedlowsmallcaps{E.T.S. Ingenierías Informática y de Telecomunicación} \\ \medskip
        
        \textit{Granada, a 18 de noviembre de 2020}

        \vfill                      

      \end{center}  
    \end{addmargin}       
  \end{titlepage}}
%\usepackage{wallpaper}



% -------------------------------------------------------------------
% Glosario
% -------------------------------------------------------------------

\documentclass{article}
\usepackage[utf8]{inputenc}
\usepackage[xindy]{glossaries}

\makeglossaries

\newglossaryentry{latex}
{
    name=latex,
    description={Is a mark up language specially suited 
    for scientific documents}
}

\newglossaryentry{maths}
{
    name=mathematics,
    description={Mathematics is what mathematicians do}
}

\glsaddall

\begin{document}
	Test.
	
	\printglossaries	
\end{document}

\begin{document}

\captionsetup[figure]{margin=1.5cm,font=small,labelfont={bf},name={Figura},labelsep=colon,textfont={it}}
\captionsetup[table]{margin=1.5cm,font=small,labelfont={bf},name={Tabla},labelsep=colon,textfont={it}}


\begin{center}
\thispagestyle{empty}
{\LARGE Universidad de Granada}\\[.5cm]
{\Large \textbf{Trabajo Fin de Grado}}\\[1cm]
\includegraphics[width=6.5cm]{img/UGR.png}\\[1cm]
{\Large Plataforma piloto para la gestión institucionalizada de los servicios de
	Relaciones Internacionales de la Universidad de Granada\\~\\}
{\huge \bfseries twinX}\\[1.5cm]
\linespread{1}
\vspace{\fill}
\includegraphics[width=2cm]{img/LSI.png}\hspace{1cm}
\includegraphics[width=3.5cm]{img/ETSIIT.png}\\[1cm]
{\Large Claudio López Carrascosa}\\[0.5cm]
{\Large Tutora: Rosana Montes Soldado}\\[0.5cm]
{\Large Departamento de Lenguajes y Sistemas Informáticos}\\[0.5cm]
{\Large de noviembre, 2020}
\end{center}

\clearpage
\pagenumbering{arabic}

% -------------------------------------------------------------------
% Contents, list of figures, list of tables
% -------------------------------------------------------------------
\newpage

% -------------------------------------------------------------------
% Main sections (as required)
% -------------------------------------------------------------------

% Primeras secciones

\section*{Autorización}

\section*{Agradecimientos}

\newpage

\setcounter{tocdepth}{4}
\setcounter{secnumdepth}{4}
\tableofcontents


\newpage

\listoffigures
\listoftables

\newpage

% 1. Introducción

\input{section1.tex}

% 2. Caso de estudio

\section{Caso de estudio: Oficina de Relaciones Internacionales de la Facultad de Filosofía y Letras de la Universidad de Granada}
\subsection{Sobre la oficina}
La labor principal de la oficina es encargarse de la gestión de los programas de intercambio y la movilidad de los estudiantes e incluso de profesores. Como no podía ser de otra manera, proporcionan la información necesaria para los interesados en estos programas, tanto de fuera de la UGR como desde universidades extranjeras. Es más, también revisan convenios existentes con éstas y hacen otros nuevos para ofrecer cada vez más alternativas para poder mejorar nuestra formación.

En su día a día, atienden peticiones y dudas de los estudiantes que participan en alguno de los citados programas; es más, se dedican a asesorar y mostrar todas las alternativas de las que disponen cuando se nos presenta alguna situación complicada, de modo que podamos resolverlo de la forma en que más nos beneficie. Trabajan, en definitiva, con el futuro de los estudiantes, pues la movilidad la desarrollamos con el objetivo de complementar nuestra formación, algo fundamental y abrumador al mismo tiempo cuando se cruzan fronteras y se quiere seguir en el camino de la educación en una universidad que no es la de casa.

La oficina se sitúa junto a la secretaría, en la Facultad de Filosofía y Letras de la Universidad de Granada, en el Campus de Cartuja, y en ella trabajan alrededor de cuatro personas. Es, dadas las cifras que se tienen, una gran cantidad de información las que tan sólo unas pocas personas tienen que manejar con una herramienta que ha sido creada sobre la marcha para facilitar su importante labor; un trabajo que no puede parar ni tolera fallos, pues los estudiantes de movilidad son uno de los pilares fundamentales de la institución.

\subsection{Servicio al estudiantado}
\subsubsection{Estudiantes salientes o \textit{outcoming}}

En relación con los estudiantes salientes, en la oficina se encargan de coordinar a los tutores académicos, que son quienes revisan los acuerdos de estudios que los candidatos proponen para iniciar su movilidad. Una vez éstos les han dado el visto bueno, en la oficina revisan cada uno de los mismos para asegurarse de que todo está en orden. Una vez iniciada la movilidad puede darse el caso en que el/la estudiante desee hacer alguna modificación a su acuerdo, debido a algún cambio imprevisto o a que alguna asignatura no fuera como se esperaba, todo ello en el destino. En ese caso, el proceso sería el mismo: tendrían que volverse a revisar de nuevo los documentos para comprobar que todo está en orden.

También escuchan casos de estudiantes con problemas particulares y que deben ser examinados con detenimiento, para ofrecer la mejor alternativa, ya sea hablando con los coordinadores de los destinos internacionales o arreglando algún dato en los convenios que haya causado algún inconveniente en la movilidad de algún(a) estudiante. De este modo, las futuras movilidades podrán hacerse con una mayor posibilidad de éxito, sobre todo cuando se trate de convenios nuevos. Son múltiples los casos en que se deba necesitar asistencia: una lengua de impartición de las asignaturas distinta a la esperada, un nivel requerido en un idioma que no se había comunicado al estudiante, etc.

Una vez el/la estudiante vuelve de su movilidad, se inicia el proceso de reconocimiento de créditos, para el cual se establecen unas correspondencias entre las calificaciones obtenidas en el destino y las que se van a especificar en su expediente en la UGR. Para ello, esta información debe estar reflejada en un documento oficial y que la secretaría pueda aceptar, por lo que en muchos casos el personal tiene que ponerse en contacto con los responsables en el destino y solicitar los certificados que sean precisos. De esta manera, se puede tener la certeza de que es alguien de confianza quien los remite, ya que se ha tomar muy en serio la veracidad de los mismos.

\subsubsection{Estudiantes entrantes o \textit{incoming}}

En cuanto a los \textit{incoming}, el proceso es distinto. Si bien es verdad que se les atiende para problemas similares a los que los estudiantes salientes podrían tener, este grupo viene a la UGR con un acuerdo de estudios previo ya hecho, de modo que es entonces cuando precisan del visto bueno extra de la Oficina, quien les confirma que las asignaturas a las que quieren acceder según lo que establezcan sus acuerdos de estudios tienen plazas disponibles. Es entonces cuando se podrían matricular de las mismas.

Este proceso es conocido como \textit{alteración de matrícula} en el ámbito de la movilidad y se hace de manera manual según las plazas que establece la facultad para cada asignatura. Es gracias a la ayuda de TWINS que puedan no sólo ver estas asignaciones de una mejor manera, sino que también tienen la posibilidad de generar los horarios para los estudiantes, algo fundamental y que les preocupa mucho cuando vienen a hacer su movilidad a la UGR. Pensemos que el simple hecho de que dos asignaturas tengan lugar en la misma hora supone un cambio inminente. Para ello, los interesados tienen que estudiar de qué alternativas disponen, atendiendo al número de plazas restantes en las demás asignaturas, no dejando de lado si la franja horaria en la que se imparte clase es compatible con su horario final.

Como es lógico, tendrán que reportar estos cambios que hagan a sus universidades de destino tal y como éstas establezcan, pues al fin y al cabo el proceso para ellos será el mismo por lo general cuando vuelvan a casa.


\subsection{Servicios al profesorado (PDI)}





\subsection{Coordinación con la Oficina de Relaciones Internacionales (ORI) de la UGR}

Todo comienza cuando la ORI envía los datos de las adjudicaciones de las plazas de movilidad a la secretaría de la facultad. Es entonces cuando comienzan los trámites administrativos: se registra a cada estudiante de acuerdo a su destino para comenzar a confeccionar su expediente en base a su acuerdo de estudios, documentos firmados y demás información necesaria. Todo ello tendrá que quedar en conocimiento de la ORI una vez acabe la movilidad.

Establecen una estrecha comunicación también cuando se tratan asuntos económicos en relación a las becas. Con la confirmación de las fechas de llegada al destino y vuelta al origen se hace un contraste con la información presente en el convenio, que es otro acuerdo que el/la estudiante se compromete a cumplir. Se tiene en cuenta si el interesado/a ha realizado la movilidad durante la totalidad del periodo para la cual estaba prevista. De no ser así, la cantidad económica final tendrá que ser distinta a la prefijada para la ayuda a recibir por el/la estudiante.

Por tanto, es de gran importancia guardar toda la información referente al proceso, pues al fin y al cabo la ORI tiene que coordinar que los distintos programas se están llevando a cabo sin incidencias, ya que, al fin y al cabo, es otro organismo asegurador del buen funcionamiento de esta alternativa al estudio continuado en la universidad que tanta importancia tiene hoy en día y que cada vez está más en auge.

\subsection{La base de datos TWINS}

%TWINS alberga actualmente unos 2055 registros de estudiantes desde el curso 2018-2019, incluyendo a algunos de ellos que han manifestado su intención de realizar un programa de movilidad en la Facultad de Filosofía y Letras durante el curso 2020/2021. 

TWINS alberga desde el curso 2018/2019 un volumen de datos que plasmamos en la tabla \ref{estadisticasTWINS}. Al tratarse de una base de datos en la herramienta ofimática Microsoft Access\textregistered, la aplicación consta más bien de una simple capa a modo de interfaz que permite al usuario interactuar con la base de datos. La presentación de los datos se posibilita gracias a la ejecución de consultas preestablecidas que se almacenan y se indica en qué campos ha de mostrarse la información. Cuando se realiza alguna acción que requiera el borrado, inserción o actualización de registros se hace uso de las macros, que son trozos de código que disponen distintos flujos de información entre las tablas implicadas en dicha operación.

\begin{figure}
	\includegraphics[width=\textwidth]{img/Capturas de TWINS/vistaConvenios.png}
	\caption{Vista de Convenios}
	\label{vistaConvenios}
\end{figure}

\begin{table}[h]
	\begin{center}
		\begin{tabular}{ | c | c | } 
			\hline
			\multicolumn{2}{|c|}{\textbf{Administración}} \\
			\hline
			Estudiantes \footnotemark & 2055 \\ 
			\hline
			Tutores & 58 \\
			\hline
			Convenios  & 607 \\ 
			\hline
			Expedientes & 5049 \\ 
			\hline
			\multicolumn{2}{|c|}{\textbf{Base de datos}} \\
			\hline
			Tablas & 108 \\
			\hline
			Relaciones & 60 \\
			\hline
			Macros & 55 \\
			\hline
			Formularios & 152 \\
			\hline
		\end{tabular}
		\caption{Estadísticas de TWINS}
		\label{estadisticasTWINS}
	\end{center}
\end{table}~
\footnotetext{Tanto \textit{incoming} como \textit{outcoming}}



\subsubsection{El modelo de datos}

El modelo relacional propuesto por Codd es el elegido para el diseño de esta base de datos. Las distintas tablas de la misma se conectan mediante relaciones que establecen restricciones para mantener la consistencia entre los datos.

Las tablas son una forma característica de almacenamiento de este modelo, en contraposición con otras disposiciones de la información usadas por otros sistemas y que, al fin y al cabo, se diseñan de otro modo porque se han de satisfacer unas necesidades distintas.

En este caso, podemos entender que la herramienta TWINS fue creada en Access y, por tanto, en un modelo de datos relacional dado el fácil acceso al usuario no experto en la materia, facilitando la operatividad con las bases de datos.

Es, sin duda, el modelo que más se utiliza aún hoy en día, el cual promete una determinada efectividad siempre y cuando el volumen de información a manejar no sea excesivamente elevado. En este caso, aunque la información que se requiere manipular en la oficina no es fácilmente manejable por personas, sí que aún podemos continuar utilizando este modelo para el desarrollo de la nueva herramienta twinX. Se entiende que no se tendrán más de unos 1000 estudiantes por curso académico (en una sola facultad) y que el número de convenios y tutores no será ni mucho menos parecido, considerando, eso sí, que la cantidad de \glspl{ExpedienteTWINS} será aproximadamente el doble que la de los estudiantes para los que se creen dichos registros.

Así pues, aparentemente, para las funcionalidades básicas y necesarias para trabajar en la oficina que twinX pretende implementar, el modelo relacional es suficiente.

No obstante, no olvidemos la intención de extender la funcionalidad del actual TWINS para que los propios estudiantes puedan dejar atrás el constante envío de documentos entre sus tutores académicos por correo electrónico, de modo que puedan dar el salto a una plataforma que implemente una interfaz que les permita prescindir de estos documentos, como el acuerdo de estudios, y trabajar con la información directamente (aunque se posibiliten las conversiones a certificados que puedan ser impresos). En este contexto, se ha de tener en cuenta la forma de tratar los datos que se quiere realizar. Si todas las asignaturas de la universidad de origen se tienen en memoria y tan sólo se tienen que registrar las de la universidad de destino, tan sólo tendríamos que guardar códigos, órdenes y los nuevos nombres para esas asignaturas. Sin embargo, si esto se complicara y tuviéramos que almacenar documentos, quizás, para esa parte de la aplicación, sería conveniente estudiar la posibilidad de implementar una base de datos cuyo modelo de datos esté enfocado a la búsqueda de estos y, no menos importante, optimizado para dicho propósito, algo que sería impensable en una base de datos relacional. Un ejemplo de estas bases de datos es \textbf{Elasticsearch}, que permite hacer búsquedas complejas en texto, a través de datos estructurados y desestructurados. Esta podría ser una buena opción a utilizar una vez se alcance una determinada complejidad en el sistema de almacenamiento, pero en nuestro caso, solo lo comentamos como posibles trabajos futuros.

\subsection{Funciones implementadas}

\subsubsection{Funcionalidades básicas}
Entre las funciones que hacen que TWINS cobre sentido, destacamos las siguientes:

\begin{itemize}
	\item \textbf{Almacenamiento de información administrativa}: estudiantes, tutores, convenios, expedientes, etc.
	\item \textbf{Asociación de estudiantes con otras entidades}: estudiantes con su tutor, el convenio respecto del cual realiza su movilidad, sus expedientes, etc.
	\item \textbf{\gls{AM}}: para poder organizar a los estudiantes \textit{incoming}
	\item \textbf{Envío masivo de correos electrónicos}: posibilita funciones como las \glspl{Nominacion} automáticas
	\item \textbf{Generación de documentos automática}: disposición de los distintos datos en documentos que son entregables a estudiantes para mostrarles un sumario de su situación académica
	\item \textbf{Creación, edición y borrado de los datos dentro de la aplicación}
\end{itemize}

\subsection{La interfaz de usuario}

\section{Estudio de necesidades}


% 3. twinX

\section{twinX}

Una vez hemos analizado el material ya existente y desde el que partiremos para la construcción de twinX, vamos a ponernos manos a la obra con su desarrollo. Si bien es cierto que no se parte desde cero en cuanto a requisitos, sí que se va a reconstruir la herramienta desde cero, para poder aportar un gran significado a todos los módulos que vayan a formar parte de twinX y así podamos cohesionarlos de una mejor manera.

\subsection{Desarrollo de twinX}

Vamos a comenzar estableciendo las ideas principales del proyecto, reafirmando el propósito y viéndolo desde otras perspectivas que, aunque parecen obvias, no siempre se tienen en cuenta. Ello nos permitirá asentar las bases del producto software final y dotarlo de calidad.

\subsubsection{Diseño centrado en el usuario}

Esta disciplina, conocida también como \textit{User-Centered Design} o UCD, destaca por basar las fases del proceso de diseño estableciendo un enfoque constante en aprender del sujeto que utilizará el producto final. Es decir, para la consecución de los objetivos, es necesario tener una retroalimentación constante por parte del usuario final, que será quien vaya orientando nuestros avances según sea su interacción con lo que vayamos desarrollando.

El proceso dispone de unas fases a seguir, como son:
\begin{itemize}
	\item \textbf{Especificación del contexto de uso:} quiénes usarán el producto, para qué y bajo qué condiciones lo harán.
	\item \textbf{Especificación de requisitos:} identificar los objetivos que tienen que cumplirse para dejar a los usuarios satisfechos.
	\item \textbf{Crear soluciones de diseño:} en distintas etapas, desde un concepto poco definido hasta un diseño completo.
	\item \textbf{Evaluación de los diseños:} a través de las pruebas con los usuarios, de forma ideal.
\end{itemize}

\paragraph{Pautas de Accesibilidad}
\leavevmode\\[\baselineskip]

El no atender a causas de accesibilidad sería no aplicar correctamente, de alguna forma, el tipo de diseño escogido para el desarrollo del producto. Cuando se hacen pruebas, el objetivo no es otro que adaptar el producto para que pueda ser bien utilizado por el mayor número de usuarios posible y de la forma satisfactoria para ellos. Es por ello por lo que tenemos que abrir el abanico y contemplar que pueden ser numerosos los usuarios que necesiten adaptaciones para interaccionar correctamente con el contenido web.

Para ello, se propone la utilización de los recursos especificados en \textit{Web Content Accessibility Guidelines} (WCAG) \citep{wcag}. Con ello, podemos hacer que la experiencia en el sitio web pueda ser mínimamente satisfactoria para todo usuario que necesite hacer uso de la misma. Para ello, se deberán emplear una serie de técnicas, especificadas en su web, para que twinX sea adaptable. Entre ellas, destacamos la posibilidad de interacción con el teclado, un texto descriptivo para las posibles imágenes que se puedan incluir, regular el contraste, etc. Con este proyecto, la intención es alcanzar el nivel AA (segundo más exigente), para que la gran mayoría de personas con necesidades especiales puedan usar la plataforma sin problema alguno.

\paragraph{Encuesta SUS}
\leavevmode\\[\baselineskip]

\subsubsection{\textit{Design Thinking}, DT}

El « Pensamiento de Diseño » o \textit{Design thinking} es una técnica de desarrollo que se centra en el usuario, pudiendo detectar y reaccionar ante cambios repentinos en el entorno de los usuarios y sus comportamientos. El objetivo está, mayormente, en abordar problemas con una pobre definición o que no se conocen a fondo para situar al usuario en el centro de todo y poder así enfocar el problema desde otras perspectivas, de manera que se pueda poner la atención en aquello que resulte de mayor importancia para los usuarios.

\paragraph{Las fases del DT}
\leavevmode\\[\baselineskip]

El proceso tiene unas fases no necesariamente secuenciales, de modo que puedan adaptarse lo mejor posible al proyecto, teniendo incluso la posibilidad de ejecutarse al mismo tiempo.

\begin{itemize}
	\item \textbf{Empatizar}, tratar de adoptar un conocimiento lo más empático posible del problema que se pretende resolver. Este es un elemento esencial, pues posibilita a los desarrolladores a descartar sus propias primeras conclusiones --erróneas a menudo-- y a entrar en materia con la realidad del cliente y sus necesidades.
	
	\item \textbf{Definir} las necesidades de los usuarios y sus problemas. Es la fase donde se reúnen y ordenan los elementos obtenidos como resultado de la fase anterior. A partir de entonces, se sintetizan para definir los problemas esenciales que se identifican, los cuales dan pie a la creación de \textit{personas}; esto es, la construcción de perfiles humanos en los cuales centrar el desarrollo.
	
	\item \textbf{Idear} y hacer frente a lo que se da por hecho, creando formas alternativas de ver y tratar el problema con soluciones innovadoras, a partir de lo estudiado en las dos fases anteriores.
	
	\item \textbf{Prototipado} de las soluciones pensadas, una fase experimental cuyo objetivo es el de encontrar la mejor solución para cada uno de los problemas encontrados. Los desarrolladores han de producir una versión de bajo coste del producto para investigar cuál es el resultado de haber llevado las ideas a la práctica.
	
	\item \textbf{Pruebas} con lo obtenido, para analizar si realmente se ha llegado a un buen resultado o, si por el contrario, se ha de retroceder a otra fase para redefinir uno o más problemas.
	
\end{itemize}

Como hemos indicado, las fases no siempre siguen el mismo orden. Hay veces que se toman decisiones como la de saltar de la primera fase de empatía a la penúltima de prototipado, probablemente para aclarar las ideas y poder hacer una mejor definición, a través de la muestra de material al cliente que pueda animarlo a dar una mayor retroalimentación. Del mismo modo, si el prototipado no ha ido bien, se puede volver a la tercera y anterior fase, la de construcción de ideas. Incluso podría darse el caso que haciendo pruebas, los desarrolladores nos demos cuenta de que no se ha llevado a cabo una buena ejecución del proceso y sea preciso volver a la segunda etapa de definición de los problemas. Siempre es mejor ir hacia atrás en lugar de comenzar la casa por el tejado, así que toda maniobra que sea apropiada para una mejor construcción del producto y que lo dote de calidad siempre es bienvenida.


\paragraph{Herramientas del DT}
\leavevmode\\[\baselineskip]

Hay una serie de actividades o técnicas que nos pueden resultar útiles para llevar a cabo el trabajo de desarrollo con eficacia y que suelen tener éxito. Destacamos las siguientes:

\begin{itemize}
	\item \textbf{Creación de personas:} perfiles ficticios de los distintos usuarios que utilizarán el producto software. Creadas en la fase de definición, no solo se especifica el propósito específico de interacción con el producto a mejorar o la necesidad por que exista el software que se quiere desarrollar. Describimos el contexto del personaje, sus inquietudes y, en definitiva, lo que hay detrás de esa persona en más ámbitos que puedan ayudar a comprender por qué es necesario que se tengan en cuenta ciertas cosas a la hora del desarrollo.
	
	\item \textbf{\textit{Brainstorming}:} también conocida como nube de ideas que radican alrededor de un concepto central. Se trata de escribir conceptos que vengan a la cabeza de los intervinientes en la creación del esquema, sin importar los análisis o el futuro que puedan tener en el proceso. Cualquier cosa que tenga que ver con lo que se está tratando es válida, pues lo que aparentemente resulta absurdo podría en muchos casos resolver parcialmente el problema o ayudar a enfocarlo. Así, cuantas más propuestas, mejor se lleva a cabo este proceso, que tiene lugar en la fase de ideación.
	
	\item \textbf{Prototipado en papel:} la creación de prototipos (en la cuarta fase de prototipado) con un material del que todos disponemos es extremadamente sencilla a la par que útil. Cuando las cosas se plasman en un folio, podemos apreciar matices que no nos venían a la cabeza cuando la idea era tan solo un concepto. Si bien es cierto que depende de las dotes artísticas de la persona que dibuja el prototipo, el hacerlo con papel y lápiz ayuda a volcar la concentración de una manera diferente a como se hace cuando se programa o se diseña con herramientas informáticas.
	
	\item \textbf{Mapas de experiencia de usuario:} también conocidos como \textit{Customer Journey Maps}, sirven para representar la experiencia de un usuario a lo largo del tiempo. En ellos plasmamos la forma en que un diseño cubre o no las necesidades de un usuario utilizando un producto o servicio. Es justo por eso por lo que estos mapas han de ser lo más descriptivos posibles, representando con gran detalle las acciones y subtareas que tiene que desempeñar un usuario al usar el sistema.
	
\end{itemize}

\subsubsection{Aplicación de las metodologías}

En relación con las fases del DT, podemos diferenciar a través de las cuales ya hemos ido pasando. El comienzo del proyecto vino acompañado de una serie de reuniones que se incluyen en los anexos

%%%%%% SECCIÓN 3.5
%\subsection{Descripción de la propuesta}
%
%Para desarrollar twinX primero se ha de elegir una metodología. En este caso, dado que este proyecto es tan sólo un comienzo para lo que sería el producto final y que tiene unos requisitos poco cerrados, escogeremos una metodología de desarrollo ágil.
%
%Este tipo de metodologías se basan en la adaptación a medios con necesidades cambiantes, con un desarrollo muy de cerca con el cliente, quien está presente junto con el equipo de desarrollo en las numerosas reuniones que se hacen a lo largo de la vida del proyecto. En este caso, el equipo de desarrollo es ajustado, pero no por ello dejaremos de lado la necesaria intervención de dos personas que hacen el papel de clientes y que son:
%
%\begin{itemize}
%	\item María Consuelo Pérez Ocaña (Coordinadora de Internacionalización en la la Facultad de Filosofía y Letras)
%	\item Miguel Ángel Sanz Sáez (personal de secretaría y creador de TWINS)
%\end{itemize}


% -------------------------------------------------------------------
% Bibliography
% -------------------------------------------------------------------

\newpage

\bibliographystyle{plainnat}
\bibliography{bibliografia} 
\nocite{*}
% -------------------------------------------------------------------
% Appendices
% -------------------------------------------------------------------

\newpage

\printglossaries

\newpage

\section*{Anexos}
\leavevmode
\begin{appendices}
	\section{Reunión de toma de contacto}
		\textit{18 de febrero de 2020}\\

		\textit{Facultad de Filosofía y Letras, Campus de Cartuja -- Universidad de Granada}\\
		
		Chelo Pérez presenta al resto del personal que trabaja en la Oficina de Relaciones Internacionales de la Facultad de Filosofía y Letras, a quienes pone a disposición para que se les haga consultas de cualquier tipo que tengan relación con el problema, pues todos conocen las dificultades a las que se han de enfrentar cada día. A continuación, Chelo presenta a Miguel Ángel Sanz, quien trabaja en la secretaría de la facultad. El interés en conocerlo se basa en que es quien ha diseñado una herramienta que resuelve parcialmente algunos de los problemas a los que se tienen que enfrentar a diario en la oficina mediante la creación de una base de datos en Access\textregistered.
		
		A pesar de su interés en la reutilización de esta herramienta que tras un año de esfuerzos está posibilitando gestionar de mejor manera la información en la oficina, se les comunica que el proyecto a construir empezaría desde cero, pudiendo así adaptarlo a otras tecnologías que facilitarían sus labores diarias considerablemente, más aún de lo que la actual solución lo hace. Es más, cabe destacar la imposibilidad de utilizar un software como es Microsoft Office\textregistered \ en un proyecto como este, dada la privatización de la licencia y las restricciones que éste pueda poner al desarrollo, que aún no se conocen, pero siempre serán menores si no se tiene fijada una herramienta sobre la cual tenga que radicar el resto del sistema.
		
		\section{Reunión de preparación del proyecto}
		
		\textit{17 de marzo de 2020}\\
		
		\textit{Google Meet}\\
		
		\textit{Participan Chelo Pérez Ocaña, Claudio López Carrascosa y Rosana Montes Soldado}\\
		
		La reunión da comienzo comentando lo hablado hasta ahora, tomando las partes esenciales del problema como punto de partida: gestión y almacenamiento de convenios, estudiantes y sus expedientes (asuntos a tratar con la secretaría). A partir de estos temas centrales, surgen aclaraciones y matices a discutir. Algunos ya se habían hablado, pero se repetían para que Rosana pudiera tomar nota y comprender el problema mejor. Otros, aunque habían sido comentados, se resalta la importancia de los mismos y se describe la forma en que se da respuesta en TWINS.
		
		Otra cosa bastante destacable es cómo Chelo describe que al principio, el usar TWINS era un verdadero desafío, pues había muchas cosas que no funcionaban y que no habían sido del todo trasladadas a la aplicación de Access, para lo que tuvo que colaborar de forma estrecha con Miguel Ángel Sanz, su creador.
		
		Sobre TWINS se comentan las cosas que se han llegado a conseguir tras los esfuerzos hechos, como una buena gestión de los convenios con otras universidades, los eventos que dispara en cuanto a envío de correos electrónicos automatizado --unos 2000 al día--. No menos importante es, la forma en que todas estas operaciones quedan registradas para su posterior volcado en un historial, donde se almacenan debidamente todos los cambios que se ha ido haciendo a cualquier información almacenada en la base de datos.
		
		En general, Chelo resalta la gran ventaja que todo ello supone, pues por ejemplo, cuando antes necesitaban dedicar alrededor de un mes al proceso de nominaciones de estudiantes, con TWINS solo precisan de una hora (y porque se va ejecutando el proceso de envío de emails independiente a cada universidad). Es más, también se resaltan mejoras como, por ejemplo, la de poner formularios a disposición de los estudiantes e integrar la información en la aplicación directamente; justo lo que han tenido que buscar para procesar la información de todos los estudiantes de movilidad una vez comenzó la pandemia por la COVID-19.
		
		Finalmente, pasamos a tratar el objetivo de mi Trabajo Fin de Grado. Con varias ideas sobre la mesa, finalmente se habla de una conversión de TWINS a la web, puesto que sería mucho más útil en cuanto a extensibilidad a otras facultades y; es más, una vez se haga el desarrollo web (con las mejoras que ya ello tiene de partida), sería más fácil integrar más servicios y añadir características. Para la realización, aparte del compromiso de Chelo para con su disponibilidad y colaboración con el trabajo, comenta la posibilidad de ceder acceso a TWINS para poder examinarlo bien y hacer un análisis completo, algo que resulta esencial para poder llevar a cabo el proyecto.
		
		\section{Reunión de presentación de TWINS}
		
		\textit{7 de abril de 2020}\\
		
		\textit{Google Meet}\\
		
		\textit{Participan Claudio López Carrascosa, Miguel Ángel Sanz Sáez y Rosana Montes Soldado}
		
		Comienza la reunión. Miguel Ángel hace uso de la palabra para hacer una pequeña presentación de TWINS. Nos explica que su nombre vino dado tras el encargo a sus hijas gemelas del diseño del logo, quienes tras explicarles el significado de los colores rojo (estudiantes salientes) y azul (estudiantes entrantes) en el ámbito de la oficina, dieron lugar a lo que es hoy por hoy el logotipo de la aplicación. Está compuesto por la palabra «gemelas» en lengua inglesa, rotulada con los dos colores. De ambos extremos de la palabra salen dos lazos que se unen en una especie de corazón morado, simbolizando la unión entre dos universidades.
		
		Acto seguido, nos comenta los objetivos de TWINS y por qué decidieron comenzar este proyecto: para manejar grandes cantidades de información, para la comunicación entre distintos actores y para lleva un correcto seguimiento de los casos de los estudiantes.
		
		También destaca aspectos de la aplicación, como el filtro de búsqueda, de suma importancia para poder localizar con eficacia a los estudiantes y/o documentos que se precisen; algo tan sencillo que ya, sin más, adelanta mucho trabajo.
		
		Terminada la presentación, comienza a hacer una demostración en directo de TWINS. Nos habla de su funcionamiento, su estructura, y nos describe todos los elementos que lo compone y que ha mencionado en la presentación.
		
		Tras aproximadamente dos horas de presentación, concluye la reunión, tras haber ido por muchos de los detalles de la aplicación (aunque no todos, por lo que probablemente se precise de más reuniones). Con todo ello, puede hacerse un análisis y así arrancar el proyecto de twinX.
		
		
\end{appendices}



\end{document}
