\section{Introducción}

Desde hace muchos años, las herramientas que hay para gestionar la información en oficinas y secretarías no son -por suerte- las mismas que antes. Y es que, por pequeña que sea la cantidad de información a manejar, hay cada día más y más formas de hacer las cosas mejor y teniendo en cuenta detalles y elementos que intervienen en la actualización haciendo, por lo general, el proceso más efectivo, eficiente y seguro.

Con este trabajo pretendemos dar un empujón a una solución que fue creada para hacer gestiones, pues estas han ido poco a poco tomando forma de problema. Cada vez hay más información que almacenar, se tiene que disponer de la misma de manera más efectiva y, en general, las exigencias van elevándose conforme pasa el tiempo. Por ello, con este trabajo, vamos a proponer una alternativa a modo de punto de partida, no solo para un ámbito en concreto, sino para muchas otras situaciones en que se tengan preámbulos parecidos al que describiremos. 

\subsection{Motivación}

La Facultad de Filosofía y Letras es, con 14 enseñanzas de grado y 2000 asignaturas, una de las facultades con mayor volumen de información de la Universidad de Granada. En el curso 2018/2019 se encuentra en cuarto lugar en cantidad de estudiantes con un total de 4733 \cite{MemAcademica}. Es más, se tiene un registro de unos 517 estudiantes entrantes y otros 236 salientes hacia y desde España, haciendo algún programa de movilidad durante el curso 2019/2020 en dicha facultad \cite{MemFYL}.

Resulta obvio pensar en que la relación del tamaño de una entidad con la cantidad de datos que maneja es directamente proporcional. Centrándonos en los datos relacionados a todos los estudiantes que realizan algún programa de movilidad, sería impensable hoy en día manejar semejante cantidad de información sin la ayuda de alguna herramienta que nos permitiera conocer el estado de los mismos (tanto los que vienen del extranjero como los que van a otro país). No obstante, lo cierto es que hasta hace tan solo unos años, desde la Coordinación de Movilidad de la facultad comenzaron con el tratamiento de los datos mediante anotaciones de todo tipo y en cualquier lugar que pudiera parecer seguro y de la forma más organizada posible (si cabía): desde documentos físicos hasta digitales, haciendo casi imposible la realización de numerosas tareas como son la clasificación, búsqueda y selección de estos datos.

Por ello, trataron de aproximarse a lo que comúnmente se utiliza en estos casos: un sistema de información. Hoy en día disponen de una herramienta que si bien les ha hecho reducir el tiempo de trabajo de semanas a unas pocas horas, sigue teniendo muchas limitaciones que no permiten adaptar la forma natural de interaccionar con la información y, por tanto, con el sistema que la alberga de manera total, poniendo en riesgo la consistencia de los datos, su seguridad y su accesibilidad, dificultando al mismo tiempo su integridad y su distribución.

Así, nace la idea de \textbf{twinX}. Se trata de una propuesta cuyo objetivo es, en cierto modo, replicar la funcionalidad de la herramienta actual creada para la Oficina de Relaciones Internacionales de la Facultad de Filosofía y Letras (ORI), conocida como \textbf{TWINS}. Será necesario establecer una serie de propuestas de rediseño (a nivel interno y al de la interfaz de usuario), un cambio de plataforma a la web y la ampliación de su alcance a la totalidad (o la mayoría) de usuarios que interaccionan con el sistema, entre otras cosas.

\subsection{Alcance}

La necesidad de una herramienta como la que ya ha sido creada para albergar la información en la actualidad no es una novedad. Son numerosas las facultades que tienen el mismo problema (aunque quizás no al mismo nivel de dificultad, pues posiblemente manejen una menor cantidad de datos), pero es desde luego una labor que no puede llevarse a cabo con directorios, archivos y demás elementos que complican mucho esta serie de tareas para cualquier cantidad de datos, teniendo en cuenta que el tratamiento que se da a veces no es simple, ni mucho menos.

Son varias las facultades que han mostrado interés por implementar TWINS. La Facultad de Ciencias Políticas trabaja en la actualidad con la herramienta y las facultades de Ciencias Económicas y Empresariales y de Bellas Artes están en ello también. En sus comienzos, la importación de datos era algo bastante tedioso y que llevó tiempo para su adaptación. Sin embargo, poco a poco, han notado una gran mejoría respecto a su anterior ritmo de trabajo.

No obstante, twinX no tiene un objetivo tan cerrado a nivel institucional. Se pretende hacer llegar una solución para la gestión de la mayor parte de la información referente a los procesos de movilidad a toda la Universidad de Granada, estableciendo las menores limitaciones posibles, de modo que pueda incluso llegar a usarse en otras universidades que no dispongan de los medios suficientes para organizar y disponer la información tal y como se espera hacer con la herramienta que proponemos.

Es más, algo que dota de atractivo a twinX es acercar la interacción con los datos a la mayor cantidad de usuarios posible que forman parte de los procesos de movilidad estudiantil, lo que presenta probablemente una alternativa muy razonable a las posibles soluciones que las distintas facultades de otras universidades puedan poseer para llevar a cabo sus gestiones en el mismo ámbito.

\subsection{Objetivos generales}

Este proyecto no es otra cosa que un camino hacia la posibilidad de transportar las gestiones que actualmente están teniendo lugar a una plataforma web.

Se trata de dejar atrás una aplicación enlazada a una base de datos local, todo ello realizado con Microsoft Access\textregistered y obtener a cambio una alta disponibilidad, un buen funcionamiento y una mejoría en la forma de gestionar el trabajo; es decir, se pretende llevar el actual TWINS al navegador, para que se pueda facilitar la posibilidad de acceso desde distintos sitios en lugar de tener los datos centralizados en un único archivo que tiene una infinidad de riesgos de ser borrado, modificado por error, corrompido, etc; todo ello, en aras de mejorar la labor de las distintas oficinas de relaciones internacionales de cualquier universidad.

No es más que la intención de mejorar y perfeccionar lo ya existente: poder consultar la información de una manera más rápida y efectiva, con una mayor comprensión de los menús por los que se pasa en el curso de la interacción con la aplicación, simplificar totalmente la vista, hacerla más atractiva y adaptarla a los sistemas web de hoy en día.

Para poder partir de lo que ya se tiene, se habrá de implementar al menos la misma funcionalidad de TWINS casi al completo. Dejando a un lado los procesos que se consideren redundantes o de poca utilidad, se habrá de incluir aquellos que sean de mayor importancia o representen un mayor manejo de los flujos de información con respecto a la totalidad de la aplicación y se incluirán otros o se modificarán los ya existentes para automatizar el intercambio de datos entre los mismos. Entre todos ellos, cabe mencionar:

\begin{itemize}
	\item Gestión de \glspl{Convenio} entre universidades.
	\item \glspl{AE}.
	\item Expedientes\footnote{Con « expedientes » aquí nos referimos más bien a las etapas del estudiante en su proceso de movilidad y no al expediente académico universitario que alberga sus calificaciones.} de los estudiantes y sus datos personales.
	\item Información sobre los \glspl{Tutor} y su gestión con respecto a los acuerdos de estudios.
	\item \gls{AM} para \glspl{Incoming}.
	\item Comunicaciones masivas, tanto al estudiantado como a los \glspl{Socio} (ver: \gls{Nominacion}).
\end{itemize}

Y con ello, los demás procesos derivados de estos siete módulos principales que se basan en la modificación, supresión y creación de la información que intercambiarán los distintos procesos de twinX.

Junto a esto y centrándonos en el día a día de la labor de una secretaría de internacionalización, hay un gran trabajo que hacer en cuanto a los acuerdos de estudios. Durante la confección de los mismos se intercambian muchas versiones de estos documentos entre los tutores académicos y  los estudiantes que planean su movilidad, y es por ello que proponemos junto con twinX una nueva forma de hacer esto. La idea no es otra que permitir que mediante una identificación, tanto por parte del tutor como por la del interesado/a en hacer una movilidad se pueda confeccionar un acuerdo de estudios online, requiriendo los elementos necesarios para dicho fin y pudiendo, de forma mucho más cómoda, intercambiar las versiones.

Con la idea anterior y en aras de reducir el papeleo al menor posible, una parte importante está en los estudiantes que vienen a estudiar a la UGR o estudiantes \textit{incoming}, como también se les conoce. Tienen que hacer saber a la secretaría qué asignaturas desean cursar y, tras ello, esperar la retroalimentación por parte de la oficina, quien determina si existen plazas disponibles para las asignaturas seleccionadas. Todo ello se haría mucho más fácil si los datos de cada uno de los estudiantes no tuvieran que ser importados a la aplicación, sino que ya existieran en ella misma directamente, pudiendo incluso dar indicaciones a los interesados sobre disponibilidad de plazas y derivados. En el mismo ámbito cabe mencionar los demás documentos que han de entregarse para justificar el fin de estancia en un país de destino, la modificación de los acuerdos de estudios o el \gls{Consentimiento}: todo ello se vería reducido a la interacción con la plataforma twinX, ofreciendo a los usuarios una visión general y compacta del curso de los eventos que marcan los procesos de movilidad.

Es más, una de las tareas que más tiempo requiere es la localización de los distintos usuarios que registra TWINS y que se han puesto en contacto con la oficina, de modo que los moderadores que administran la información de los estudiantes tienen que buscar en distintos sitios según sea la naturaleza de la petición que tengan que atender. Por ello, y dado que el correo electrónico --dando por hecho que éste es el medio de comunicación habitual-- es un servicio externo de la aplicación, proponemos una unificación para simplificar no sólo el curso de las operaciones que el usuario tenga que desempeñar, sino  también la consistencia de la información y la visualización de la misma, poniendo ante la vista una perspectiva más genérica aún del estudiante, lo que terminaría haciendo más efectivo el trabajo.


Todo ello ha de estar respaldado por una capacidad de la plataforma de albergar una cierta cantidad de conexiones al mismo tiempo. Si bien es verdad que no se contempla soportar una elevada carga de peticiones simultáneas debido a que los accesos se condensarán al comienzo y al final de un cuatrimestre académico, se ha de tener en cuenta que se ha de servir la plataforma a tantos estudiantes como deseen acceder a la misma.