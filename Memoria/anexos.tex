\appendix
\chapter{Reuniones con el cliente}
	\section{Reunión de toma de contacto}
	\label{reunion1}
		\textit{18 de febrero de 2020}\\
		
		\textit{Duración: 45 minutos}\\

		\textit{Facultad de Filosofía y Letras, Campus de Cartuja -- Universidad de Granada}\\
		
		Chelo Pérez presenta al resto del personal que trabaja en la Oficina de Relaciones Internacionales de la Facultad de Filosofía y Letras, a quienes pone a disposición para que se les haga consultas de cualquier tipo que tengan relación con el problema, pues todos conocen las dificultades a las que se han de enfrentar cada día. A continuación, Chelo presenta a Miguel Ángel Sanz, quien trabaja en la secretaría de la facultad. El interés en conocerlo se basa en que es quien ha diseñado una herramienta que resuelve parcialmente algunos de los problemas a los que se tienen que enfrentar a diario en la oficina mediante la creación de una base de datos en Access\textregistered.
		
		A pesar de su interés en la reutilización de esta herramienta que tras un año de esfuerzos está posibilitando gestionar de mejor manera la información en la oficina, se les comunica que el proyecto a construir empezaría desde cero, pudiendo así adaptarlo a otras tecnologías que facilitarían sus labores diarias considerablemente, más aún de lo que la actual solución lo hace. Es más, cabe destacar la imposibilidad de utilizar un software como es Microsoft Office\textregistered \ en un proyecto como este, dada la privatización de la licencia y las restricciones que éste pueda poner al desarrollo, que aún no se conocen, pero siempre serán menores si no se tiene fijada una herramienta sobre la cual tenga que radicar el resto del sistema.
		
	\section{Reunión de preparación del proyecto}
	\label{reunion2}
	
	\textit{17 de marzo de 2020}\\
	
	\textit{Duración: 34 minutos}\\
	
	\textit{A través de Google Meet}\\
	
	\textit{Participan Chelo Pérez Ocaña, Claudio López Carrascosa y Rosana Montes Soldado}\\
	
	La reunión da comienzo comentando lo hablado hasta ahora, tomando las partes esenciales del problema como punto de partida: gestión y almacenamiento de convenios, estudiantes y sus expedientes (asuntos a tratar con la secretaría). A partir de estos temas centrales, surgen aclaraciones y matices a discutir. Algunos ya se habían hablado, pero se repetían para que Rosana pudiera tomar nota y comprender el problema mejor. Otros, aunque habían sido comentados, se resalta la importancia de los mismos y se describe la forma en que se da respuesta en TWINS.
	
	Otra cosa bastante destacable es cómo Chelo describe que al principio, el usar TWINS era un verdadero desafío, pues había muchas cosas que no funcionaban y que no habían sido del todo trasladadas a la aplicación de Access, para lo que tuvo que colaborar de forma estrecha con Miguel Ángel Sanz, su creador.
	
	Sobre TWINS se comentan las cosas que se han llegado a conseguir tras los esfuerzos hechos, como una buena gestión de los convenios con otras universidades, los eventos que dispara en cuanto a envío de correos electrónicos automatizado --unos 2000 al día--. No menos importante es, la forma en que todas estas operaciones quedan registradas para su posterior volcado en un historial, donde se almacenan debidamente todos los cambios que se ha ido haciendo a cualquier información almacenada en la base de datos.
	
	En general, Chelo resalta la gran ventaja que todo ello supone, pues por ejemplo, cuando antes necesitaban dedicar alrededor de un mes al proceso de nominaciones de estudiantes, con TWINS solo precisan de una hora (y porque se va ejecutando el proceso de envío de emails independiente a cada universidad). Es más, también se resaltan mejoras como, por ejemplo, la de poner formularios a disposición de los estudiantes e integrar la información en la aplicación directamente; justo lo que han tenido que buscar para procesar la información de todos los estudiantes de movilidad una vez comenzó la pandemia por la COVID-19.
	
	Finalmente, pasamos a tratar el objetivo de mi Trabajo Fin de Grado. Con varias ideas sobre la mesa, finalmente se habla de una conversión de TWINS a la web, puesto que sería mucho más útil en cuanto a extensibilidad a otras facultades y; es más, una vez se haga el desarrollo web (con las mejoras que ya ello tiene de partida), sería más fácil integrar más servicios y añadir características. Para la realización, aparte del compromiso de Chelo para con su disponibilidad y colaboración con el trabajo, comenta la posibilidad de ceder acceso a TWINS para poder examinarlo bien y hacer un análisis completo, algo que resulta esencial para poder llevar a cabo el proyecto.
	
	\section{Reunión de presentación de TWINS}
	\label{reunion3}
	
	\textit{7 de abril de 2020}\\
	
	\textit{Duración: 1 hora y 56 minutos}\\
	
	\textit{A través de Google Meet}\\
	
	\textit{Participan Claudio López Carrascosa, Miguel Ángel Sanz Sáez y Rosana Montes Soldado}
	
	Comienza la reunión. Miguel Ángel hace uso de la palabra para hacer una pequeña presentación de TWINS. Nos explica que su nombre vino dado tras el encargo a sus hijas gemelas del diseño del logo, quienes tras explicarles el significado de los colores rojo (estudiantes salientes) y azul (estudiantes entrantes) en el ámbito de la oficina, dieron lugar a lo que es hoy por hoy el logotipo de la aplicación. Está compuesto por la palabra «gemelas» en lengua inglesa, rotulada con los dos colores. De ambos extremos de la palabra salen dos lazos que se unen en una especie de corazón morado, simbolizando la unión entre dos universidades.
	
	Acto seguido, nos comenta los objetivos de TWINS y por qué decidieron comenzar este proyecto: para manejar grandes cantidades de información, para la comunicación entre distintos actores y para lleva un correcto seguimiento de los casos de los estudiantes.
	
	También destaca aspectos de la aplicación, como el filtro de búsqueda, de suma importancia para poder localizar con eficacia a los estudiantes y/o documentos que se precisen; algo tan sencillo que ya, sin más, adelanta mucho trabajo.
	
	Terminada la presentación, comienza a hacer una demostración en directo de TWINS. Nos habla de su funcionamiento, su estructura, y nos describe todos los elementos que lo compone y que ha mencionado en la presentación.
	
	Tras aproximadamente dos horas de presentación, concluye la reunión, tras haber ido por muchos de los detalles de la aplicación (aunque no todos, por lo que probablemente se precise de más reuniones). Con todo ello, puede hacerse un análisis y así arrancar el proyecto de twinX.
		
	\section{Reunión de orientación para la instalación de TWINS}
	\label{reunion4}
	
	\textit{18 de abril de 2020}\\
	
	\textit{Duración: 18 minutos}\\
	
	\textit{A través de Google Meet}\\
	
	\textit{Participan Claudio López Carrascosa y Miguel Ángel Sanz Sáez}
	
	En esta reunión se dieron las instrucciones necesarias para la instalación de TWINS, a partir de los dos archivos necesarios para ello: por un lado, el de la aplicación; por otro, el de los datos a importar por la primera.
	
	Todo ello fue indicado por Miguel Ángel tras haber recibido firmado por parte de Rosana (tutora) y Claudio (estudiante) el correspondiente compromiso de confidencialidad para con la información de carácter sensible que alberga TWINS, si bien es cierto que ya había sido sustituida por otras cadenas de texto gran parte de la misma, imposibilitando la difusión de los datos. Sin embargo, estos cambios no supusieron ningún impedimento para el correcto análisis de la aplicación, que se centró mayormente en sus funcionalidades.
	
	\chapter{Código en DBML para la generación del modelo de la base de datos}
	\label{anexo:dbml}
	
	\lstset{breaklines=true}
	\begin{lstlisting}
		// ENTIDADES BASICAS
		
		Table centro as CEN {
			id int [pk, increment]
			nombre varchar [not null, unique]
		}
		
		Table curso as CURSO {
			id int [pk, increment]
			curso varchar(4) [not null]
		}
		
		Table mail_predef as MAIL {
			id int [pk, increment]
			titulo varchar [not null, unique]
			asunto varchar [not null]
			cuerpo text [not null]
		}
		
		Table titulacion as TIT {
			id int [pk]
			nombre varchar [not null, unique]
			id_centro int [not null]
		}
		
		Ref: TIT.id_centro > CEN.id
		
		Table asignatura as ASIG {
			id int [pk]
			id_tit int [not null]
			nombre varchar [not null]
			ects int [not null]
			cuatrimestre cuatrimestre [not null]
			tipo tipo_asignatura [not null]
		}
		
		Ref: ASIG.id_tit > TIT.id
		
		Table asignatura_ext as EXT_ASIG {
			id int [pk]
			id_conv int [not null] 
			nombre varchar [not null]
			ects int [not null]
			id_curso int [ref: > CURSO.id, not null] 
			cuatrimestre cuatrimestre [not null]
		}
		
		Ref: EXT_ASIG.id_conv > CON.id
		
		Table pais as P {
			iso varchar [pk]
			nombre varchar [not null]
		}
		
		Table universidad as UNI {
			cod_uni varchar [not null]
			cod_pais varchar [not null]
			nombre varchar [not null]
			direccion varchar 
			web varchar
			email varchar
			
			Indexes {
				(cod_pais, cod_uni) [pk]
			}
		}
		
		Ref: UNI.cod_pais > P.iso
		
		Table area as AR {
			cod_isced varchar [pk]
			nombre_isced varchar [not null]
			nombre_area varchar 
			
		}
		
		// USUARIOS Y TIPOS
		
		Table user as U {
			id int [pk, increment]
			username varchar
			nombre varchar
			tipo tipo_usuario
			password varchar
			email varchar
			telefono varchar
			genero genero
		}
		
		Table estudiante as EST{
			id_usuario int [not null]
			dni varchar [not null, unique] // PUEDE SER NIE
			id_convenio int [not null]
			id_titulacion int [not null]
			telefono2 int 
			email_go_ugr varchar 
			f_nacimiento datetime [not null]
			tipo_estudiante tipo_estudiante [not null]
			cesion_datos boolean 
			nota_expediente double 
			beca_mec boolean
		}
		
		Ref: EST.id_titulacion > TIT.id
		Ref: U.id - EST.id_usuario
		Ref: EST.id_convenio > CON.id
		
		
		// CONVENIOS Y ACUERDOS
		
		Table competencia_ling as CL {
			id int [pk, increment]
			lengua varchar [not null]
			nivel nivel_idioma [not null]
		}
		
		Table rel_cl_est {
			id int [pk, increment]
			id_cl int [ref: > CL.id, not null]
			id_est int [ref: > EST.id_usuario, not null]
		}
		
		Table req_ling_conv {
			id int [pk, increment]
			id_comp int [ref: > CL.id, not null]
			id_conv int [ref: > CON.id, not null]
		}
		
		Table convenio as CON {
			id int [pk, increment]
			cod_area varchar [not null]
			cod_uni varchar [not null]
			cod_pais varchar [not null]
			id_admon_out int
			
			id_curso_creacion int [ref: > CURSO.id, not null]
			creado_por int [not null]
			
			num_becas_in int [not null]
			num_becas_out int [not null]
			
			meses_in int [not null]
			meses_out int [not null]
			
			anno_inicio int [not null]
			anno_fin int [not null]
			
			req_titulacion varchar
			req_curso varchar
			
			nominacion_online boolean 
			link_nom_online varchar 
			info_nom_online text
			
			link_documentacion varchar
			
			movilidad_pdi boolean 
			movilidad_pas boolean 
			
			tipo_movilidad tipo_movilidad [not null]
			
			user_online varchar
			password_online varchar
			fecha_online datetime
			
			info_tor text 
			
			observ_discapacidad text
			observ_req_ling text 
			
			begin_nom_1s datetime
			end_nom_1s datetime
			begin_nom_2s datetime
			end_nom_2s datetime
			begin_app_1s datetime
			end_app_1s datetime
			begin_app_2s datetime
			end_app_2s datetime
			begin_mov_1s datetime
			end_mov_1s datetime
			begin_mov_2s datetime
			end_mov_2s datetime
			
			memo_grading text
			memo_visado text
			memo_seguro text
			memo_alojamiento text
			
			nombre_coord varchar
			cargo_coord varchar
			email_coord varchar
			tlf_coord varchar
			address_coord varchar [note:"Por defecto igual que la de la universidad"]
			web_inf_acad varchar
			
			nombre_admon_in varchar
			cargo_admon_in varchar
			mail_admon_in varchar
			nombre_resp_acad_in varchar
			cargo_resp_acad_in varchar
			
			nombre_admon_out varchar
			cargo_admon_out varchar
			mail_admon_out varchar
			nombre_resp_acad_out varchar
			cargo_resp_acad_out varchar
			mail_resp_acad_out varchar
			
		}
		
		Ref: P.iso < CON.cod_pais
		Ref: CON.creado_por > U.id
		Ref: CON.cod_area > AR.cod_isced
		Ref: CON.(cod_pais, cod_uni) > UNI.(cod_pais, cod_uni)
		
		
		Table acuerdo_estudios as AE {
			id int [pk, increment]
			id_estudiante int [not null]
			id_tutor int [not null]
			
			timestamp_creacion datetime [not null]
			periodo cuatrimestre [not null]
			fase int [not null]
			id_curso int [ref: > CURSO.id, not null]
			necesidades text
			begin_movilidad date // comienzo individual del estudiante
			end_movilidad date // fin individual del estudiante
			timestamp_nominacion datetime 
			timestamp_registro datetime [not null]
			link_documentacion varchar
			n_solicitud_RRII int
			convocatoria convocatoria [not null]
		}
		
		Ref: AE.id_estudiante > EST.id_usuario
		Ref: AE.id_tutor > U.id
		
		Table ae_asigloc_asigext {
			id_ae int
			asig_loc int [ref: > ASIG.id]
			asig_ext int [ref: > EXT_ASIG.id]
			
			Indexes {
				(id_ae, asig_loc, asig_ext) [pk]
			}
		}
		
		Table renuncia as REN{
			id int [pk, increment]
			id_ae int  [not null, ref: - AE.id]
			descripcion text [not null]
			timestamp datetime [not null]
		}
		
		// EXPEDIENTES, TIPOS DE EXPDIENTE Y FASES
		
		Table expediente as EXP{
			id int [pk, increment]
			id_ae int [not null]
			id_tipo_exp int [not null]
		}
		
		Ref: EXP.id_ae > AE.id
		Ref: EXP.id_tipo_exp > TIPO_EXP.id
		
		Table tipo_expediente as TIPO_EXP {
			id int [pk, increment]
			descripcion varchar [not null]
			tipo_estudiante tipo_estudiante [not null]
		}
		
		Table fase_expediente as FAS_EXP {
			id int [pk, increment]
			id_tipo_exp int [not null]
			descripcion varchar [not null]
			fase_final boolean [default: 0]
		}
		
		Ref: FAS_EXP.id_tipo_exp > TIPO_EXP.id
		
		Table envio_mail_fase {
			id int [pk, increment]
			id_mail int [ref: > MAIL.id, not null]
			id_fase int [ref: > FAS_EXP.id, not null]
			cargo cargo [not null]
		}
		
		Table hist_envio_mail_fase {
			id int [pk, increment]
			id_mail int [ref: > MAIL.id, not null]
			id_fase int [ref: > FAS_EXP.id, not null]
			id_exp int [ref: > EXP.id, not null]
			email varchar [not null]
		}
		
		Table hist_envio_mail_fase_mod { // Mail modificado
			id int [pk, increment]
			asunto varchar
			cuerpo text [not null]
			id_fase int [ref: > FAS_EXP.id, not null]
			id_exp int [ref: > EXP.id, not null]
			email varchar [not null]
		}
		
		Table rel_exp_fase as REL_EF {
			id int [pk]
			id_exp int [not null]
			id_fase int [not null]
			id_gestor int [not null]
			procesado boolean 
			timestamp datetime [not null]
			info varchar
		}
		
		Ref: REL_EF.id_exp - EXP.id
		Ref: REL_EF.id_fase - FAS_EXP.id
		Ref: REL_EF.id_gestor - U.id
		
		Table rel_exp_fav_gestor { 
			id int [pk, increment]
			id_exp int [ref: > EXP.id]
			id_gestor int [ref: > U.id]
			//tipo_usuario == GESTOR
		}
		
		// CALENDARIO
		
		Table evento as EV {
			id int [pk, increment]
			id_creador int [not null]
			titulo varchar [not null]
			descripcion text
			estado estado_evento_tarea [not null]
			prioridad prioridad [not null, default: 'MEDIA']
		}
		
		Ref: EV.id_creador - U.id //tipo == GESTOR
		
		Table tarea as TASK {
			id int [pk, increment]
			descripcion text [not null]
			estado estado_evento_tarea [not null, default: 'PENDIENTE']
		}
		
		Table deadline_aviso as DLA { 
			id int [pk, increment]
			fecha datetime [not null]
			id_responsable int [not null]
			id_evento int  [not null]
			id_tarea int [not null]
		}
		
		Ref: DLA.id_responsable > U.id // tipo == GESTOR
		Ref: DLA.id_evento > EV.id
		Ref: DLA.id_tarea > TASK.id
		
		Table recordatorio {
			id int [pk, increment]
			timestamp datetime [not null]
			id_usuario int [ref: > U.id, not null]
			deadline datetime [not null]
			titulo varchar [not null]
			descripcion text
			completado boolean
		}
		
		
		// MENSAJES	
		Table mensaje as MSG{
			id int [pk, increment]
			timestamp datetime [not null]
			id_emisor int [not null]
			id_receptor int [not null]
			leido boolean
			etiqueta etiqueta_msg [not null]
			asunto varchar 
			cuerpo text [not null]
			
		}
		
		Ref: MSG.id_emisor > U.id
		Ref: MSG.id_receptor > U.id
		
		Enum tipo_asignatura {
			TRONCAL
			OBLIGATORIA
			OPTATIVA
		}
		
		Enum cargo {
			COORDINADOR
			ADMON_IN
			ADMON_OUT
			RESP_ADMON_OUT
		}
		
		Enum nivel_idioma {
			B1
			B2
			C1
			C2
		}
		
		
		Enum etiqueta_msg {
			IMPORTANTE
			ELIMINADO
		}
		
		Enum estado_evento_tarea {
			PENDIENTE
			EN_PROCESO
			TERMINADO
		}
		
		Enum prioridad {
			ALTA
			MEDIA
			BAJA
		}
		
		Enum estado_ae {
			REVISION
			DENEGADO
			ACEPTADO
			VIGENTE
		}
		
		Enum convocatoria {
			PRIMERA
			SEGUNDA
			EXTRAORDINARIA
		}
		
		Enum tipo_movilidad {
			ERASMUS
			ARQUS
			ERASMUS_DI
			ERASMUS_PARTNER
			INTERCAMBIO
			LIBRE_MOVILIDAD
		}
		
		Enum tipo_usuario {
			SUPERUSUARIO
			GESTOR
			ESTUDIANTE
			TUTOR
		}
		
		Enum genero {
			F
			M
			O
		}
		
		Enum tipo_estudiante {
			INCOMING
			OUTGOING
		}
		
		Enum cuatrimestre {
			PRIMERO
			SEGUNDO
			C_COMPLETO
		}
		
	\end{lstlisting}

	\chapter{Código en MySQL generado a partir del \hyperref[anexo:dbml]{código en DBML}}
	\label{anexo:mysql}
	
	\begin{minted}[breaklines]{mysql}
		
		CREATE TABLE `centro` (
		`id` int PRIMARY KEY AUTO_INCREMENT,
		`nombre` varchar(255) UNIQUE NOT NULL
		);
		
		CREATE TABLE `curso` (
		`id` int PRIMARY KEY AUTO_INCREMENT,
		`curso` varchar(4) NOT NULL
		);
		
		CREATE TABLE `mail_predef` (
		`id` int PRIMARY KEY AUTO_INCREMENT,
		`titulo` varchar(255) UNIQUE NOT NULL,
		`asunto` varchar(255) NOT NULL,
		`cuerpo` text NOT NULL
		);
		
		CREATE TABLE `titulacion` (
		`id` int PRIMARY KEY,
		`nombre` varchar(255) UNIQUE NOT NULL,
		`id_centro` int NOT NULL
		);
		
		CREATE TABLE `asignatura` (
		`id` int PRIMARY KEY,
		`id_tit` int NOT NULL,
		`nombre` varchar(255) NOT NULL,
		`ects` int NOT NULL,
		`cuatrimestre` ENUM ('PRIMERO', 'SEGUNDO', 'C_COMPLETO') NOT NULL,
		`tipo` ENUM ('TRONCAL', 'OBLIGATORIA', 'OPTATIVA') NOT NULL
		);
		
		CREATE TABLE `asignatura_ext` (
		`id` int PRIMARY KEY,
		`id_conv` int NOT NULL,
		`nombre` varchar(255) NOT NULL,
		`ects` int NOT NULL,
		`id_curso` int NOT NULL,
		`cuatrimestre` ENUM ('PRIMERO', 'SEGUNDO', 'C_COMPLETO') NOT NULL
		);
		
		CREATE TABLE `pais` (
		`iso` varchar(255) PRIMARY KEY,
		`nombre` varchar(255) NOT NULL
		);
		
		CREATE TABLE `universidad` (
		`cod_uni` varchar(255) NOT NULL,
		`cod_pais` varchar(255) NOT NULL,
		`nombre` varchar(255) NOT NULL,
		`direccion` varchar(255),
		`web` varchar(255),
		`email` varchar(255),
		PRIMARY KEY (`cod_pais`, `cod_uni`)
		);
		
		CREATE TABLE `area` (
		`cod_isced` varchar(255) PRIMARY KEY,
		`nombre_isced` varchar(255) NOT NULL,
		`nombre_area` varchar(255)
		);
		
		CREATE TABLE `user` (
		`id` int PRIMARY KEY AUTO_INCREMENT,
		`username` varchar(255),
		`nombre` varchar(255),
		`tipo` ENUM ('SUPERUSUARIO', 'GESTOR', 'ESTUDIANTE', 'TUTOR'),
		`password` varchar(255),
		`email` varchar(255),
		`telefono` varchar(255),
		`genero` ENUM ('F', 'M', 'O')
		);
		
		CREATE TABLE `estudiante` (
		`id_usuario` int NOT NULL,
		`dni` varchar(255) UNIQUE NOT NULL,
		`id_convenio` int NOT NULL,
		`id_titulacion` int NOT NULL,
		`telefono2` int,
		`email_go_ugr` varchar(255),
		`f_nacimiento` datetime NOT NULL,
		`tipo_estudiante` ENUM ('INCOMING', 'OUTGOING') NOT NULL,
		`cesion_datos` boolean,
		`nota_expediente` double,
		`beca_mec` boolean
		);
		
		CREATE TABLE `competencia_ling` (
		`id` int PRIMARY KEY AUTO_INCREMENT,
		`lengua` varchar(255) NOT NULL,
		`nivel` ENUM ('B1', 'B2', 'C1', 'C2') NOT NULL
		);
		
		CREATE TABLE `rel_cl_est` (
		`id` int PRIMARY KEY AUTO_INCREMENT,
		`id_cl` int NOT NULL,
		`id_est` int NOT NULL
		);
		
		CREATE TABLE `req_ling_conv` (
		`id` int PRIMARY KEY AUTO_INCREMENT,
		`id_comp` int NOT NULL,
		`id_conv` int NOT NULL
		);
		
		CREATE TABLE `convenio` (
		`id` int PRIMARY KEY AUTO_INCREMENT,
		`cod_area` varchar(255) NOT NULL,
		`cod_uni` varchar(255) NOT NULL,
		`cod_pais` varchar(255) NOT NULL,
		`id_admon_out` int,
		`id_curso_creacion` int NOT NULL,
		`creado_por` int NOT NULL,
		`num_becas_in` int NOT NULL,
		`num_becas_out` int NOT NULL,
		`meses_in` int NOT NULL,
		`meses_out` int NOT NULL,
		`anno_inicio` int NOT NULL,
		`anno_fin` int NOT NULL,
		`req_titulacion` varchar(255),
		`req_curso` varchar(255),
		`nominacion_online` boolean,
		`link_nom_online` varchar(255),
		`info_nom_online` text,
		`link_documentacion` varchar(255),
		`movilidad_pdi` boolean,
		`movilidad_pas` boolean,
		`tipo_movilidad` ENUM ('ERASMUS', 'ARQUS', 'ERASMUS_DI', 'ERASMUS_PARTNER', 'INTERCAMBIO', 'LIBRE_MOVILIDAD') NOT NULL,
		`user_online` varchar(255),
		`password_online` varchar(255),
		`fecha_online` datetime,
		`info_tor` text,
		`observ_discapacidad` text,
		`observ_req_ling` text,
		`begin_nom_1s` datetime,
		`end_nom_1s` datetime,
		`begin_nom_2s` datetime,
		`end_nom_2s` datetime,
		`begin_app_1s` datetime,
		`end_app_1s` datetime,
		`begin_app_2s` datetime,
		`end_app_2s` datetime,
		`begin_mov_1s` datetime,
		`end_mov_1s` datetime,
		`begin_mov_2s` datetime,
		`end_mov_2s` datetime,
		`memo_grading` text,
		`memo_visado` text,
		`memo_seguro` text,
		`memo_alojamiento` text,
		`nombre_coord` varchar(255),
		`cargo_coord` varchar(255),
		`email_coord` varchar(255),
		`tlf_coord` varchar(255),
		`address_coord` varchar(255) COMMENT 'Por defecto igual que la de la universidad',
		`web_inf_acad` varchar(255),
		`nombre_admon_in` varchar(255),
		`cargo_admon_in` varchar(255),
		`mail_admon_in` varchar(255),
		`nombre_resp_acad_in` varchar(255),
		`cargo_resp_acad_in` varchar(255),
		`nombre_admon_out` varchar(255),
		`cargo_admon_out` varchar(255),
		`mail_admon_out` varchar(255),
		`nombre_resp_acad_out` varchar(255),
		`cargo_resp_acad_out` varchar(255),
		`mail_resp_acad_out` varchar(255)
		);
		
		CREATE TABLE `acuerdo_estudios` (
		`id` int PRIMARY KEY AUTO_INCREMENT,
		`id_estudiante` int NOT NULL,
		`id_tutor` int NOT NULL,
		`timestamp_creacion` datetime NOT NULL,
		`periodo` ENUM ('PRIMERO', 'SEGUNDO', 'C_COMPLETO') NOT NULL,
		`fase` int NOT NULL,
		`id_curso` int NOT NULL,
		`necesidades` text,
		`begin_movilidad` date,
		`end_movilidad` date,
		`timestamp_nominacion` datetime,
		`timestamp_registro` datetime NOT NULL,
		`link_documentacion` varchar(255),
		`n_solicitud_RRII` int,
		`convocatoria` ENUM ('PRIMERA', 'SEGUNDA', 'EXTRAORDINARIA') NOT NULL
		);
		
		CREATE TABLE `ae_asigloc_asigext` (
		`id_ae` int,
		`asig_loc` int,
		`asig_ext` int,
		PRIMARY KEY (`id_ae`, `asig_loc`, `asig_ext`)
		);
		
		CREATE TABLE `renuncia` (
		`id` int PRIMARY KEY AUTO_INCREMENT,
		`id_ae` int NOT NULL,
		`descripcion` text NOT NULL,
		`timestamp` datetime NOT NULL
		);
		
		CREATE TABLE `expediente` (
		`id` int PRIMARY KEY AUTO_INCREMENT,
		`id_ae` int NOT NULL,
		`id_tipo_exp` int NOT NULL
		);
		
		CREATE TABLE `tipo_expediente` (
		`id` int PRIMARY KEY AUTO_INCREMENT,
		`descripcion` varchar(255) NOT NULL,
		`tipo_estudiante` ENUM ('INCOMING', 'OUTGOING') NOT NULL
		);
		
		CREATE TABLE `fase_expediente` (
		`id` int PRIMARY KEY AUTO_INCREMENT,
		`id_tipo_exp` int NOT NULL,
		`descripcion` varchar(255) NOT NULL,
		`fase_final` boolean DEFAULT 0
		);
		
		CREATE TABLE `envio_mail_fase` (
		`id` int PRIMARY KEY AUTO_INCREMENT,
		`id_mail` int NOT NULL,
		`id_fase` int NOT NULL,
		`cargo` ENUM ('COORDINADOR', 'ADMON_IN', 'ADMON_OUT', 'RESP_ADMON_OUT') NOT NULL
		);
		
		CREATE TABLE `hist_envio_mail_fase` (
		`id` int PRIMARY KEY AUTO_INCREMENT,
		`id_mail` int NOT NULL,
		`id_fase` int NOT NULL,
		`id_exp` int NOT NULL,
		`email` varchar(255) NOT NULL
		);
		
		CREATE TABLE `hist_envio_mail_fase_mod` (
		`id` int PRIMARY KEY AUTO_INCREMENT,
		`asunto` varchar(255),
		`cuerpo` text NOT NULL,
		`id_fase` int NOT NULL,
		`id_exp` int NOT NULL,
		`email` varchar(255) NOT NULL
		);
		
		CREATE TABLE `rel_exp_fase` (
		`id` int PRIMARY KEY,
		`id_exp` int NOT NULL,
		`id_fase` int NOT NULL,
		`id_gestor` int NOT NULL,
		`procesado` boolean,
		`timestamp` datetime NOT NULL,
		`info` varchar(255)
		);
		
		CREATE TABLE `rel_exp_fav_gestor` (
		`id` int PRIMARY KEY AUTO_INCREMENT,
		`id_exp` int,
		`id_gestor` int
		);
		
		CREATE TABLE `evento` (
		`id` int PRIMARY KEY AUTO_INCREMENT,
		`id_creador` int NOT NULL,
		`titulo` varchar(255) NOT NULL,
		`descripcion` text,
		`estado` ENUM ('PENDIENTE', 'EN_PROCESO', 'TERMINADO') NOT NULL,
		`prioridad` ENUM ('ALTA', 'MEDIA', 'BAJA') NOT NULL DEFAULT "MEDIA"
		);
		
		CREATE TABLE `tarea` (
		`id` int PRIMARY KEY AUTO_INCREMENT,
		`descripcion` text NOT NULL,
		`estado` ENUM ('PENDIENTE', 'EN_PROCESO', 'TERMINADO') NOT NULL DEFAULT "PENDIENTE"
		);
		
		CREATE TABLE `deadline_aviso` (
		`id` int PRIMARY KEY AUTO_INCREMENT,
		`fecha` datetime NOT NULL,
		`id_responsable` int NOT NULL,
		`id_evento` int NOT NULL,
		`id_tarea` int NOT NULL
		);
		
		CREATE TABLE `recordatorio` (
		`id` int PRIMARY KEY AUTO_INCREMENT,
		`timestamp` datetime NOT NULL,
		`id_usuario` int NOT NULL,
		`deadline` datetime NOT NULL,
		`titulo` varchar(255) NOT NULL,
		`descripcion` text,
		`completado` boolean
		);
		
		CREATE TABLE `mensaje` (
		`id` int PRIMARY KEY AUTO_INCREMENT,
		`timestamp` datetime NOT NULL,
		`id_emisor` int NOT NULL,
		`id_receptor` int NOT NULL,
		`leido` boolean,
		`etiqueta` ENUM ('IMPORTANTE', 'ELIMINADO') NOT NULL,
		`asunto` varchar(255),
		`cuerpo` text NOT NULL
		);
		
		ALTER TABLE `titulacion` ADD FOREIGN KEY (`id_centro`) REFERENCES `centro` (`id`);
		
		ALTER TABLE `asignatura` ADD FOREIGN KEY (`id_tit`) REFERENCES `titulacion` (`id`);
		
		ALTER TABLE `asignatura_ext` ADD FOREIGN KEY (`id_curso`) REFERENCES `curso` (`id`);
		
		ALTER TABLE `asignatura_ext` ADD FOREIGN KEY (`id_conv`) REFERENCES `convenio` (`id`);
		
		ALTER TABLE `universidad` ADD FOREIGN KEY (`cod_pais`) REFERENCES `pais` (`iso`);
		
		ALTER TABLE `estudiante` ADD FOREIGN KEY (`id_titulacion`) REFERENCES `titulacion` (`id`);
		
		ALTER TABLE `estudiante` ADD FOREIGN KEY (`id_usuario`) REFERENCES `user` (`id`);
		
		ALTER TABLE `estudiante` ADD FOREIGN KEY (`id_convenio`) REFERENCES `convenio` (`id`);
		
		ALTER TABLE `rel_cl_est` ADD FOREIGN KEY (`id_cl`) REFERENCES `competencia_ling` (`id`);
		
		ALTER TABLE `rel_cl_est` ADD FOREIGN KEY (`id_est`) REFERENCES `estudiante` (`id_usuario`);
		
		ALTER TABLE `req_ling_conv` ADD FOREIGN KEY (`id_comp`) REFERENCES `competencia_ling` (`id`);
		
		ALTER TABLE `req_ling_conv` ADD FOREIGN KEY (`id_conv`) REFERENCES `convenio` (`id`);
		
		ALTER TABLE `convenio` ADD FOREIGN KEY (`id_curso_creacion`) REFERENCES `curso` (`id`);
		
		ALTER TABLE `convenio` ADD FOREIGN KEY (`cod_pais`) REFERENCES `pais` (`iso`);
		
		ALTER TABLE `convenio` ADD FOREIGN KEY (`creado_por`) REFERENCES `user` (`id`);
		
		ALTER TABLE `convenio` ADD FOREIGN KEY (`cod_area`) REFERENCES `area` (`cod_isced`);
		
		ALTER TABLE `convenio` ADD FOREIGN KEY (`cod_pais`, `cod_uni`) REFERENCES `universidad` (`cod_pais`, `cod_uni`);
		
		ALTER TABLE `acuerdo_estudios` ADD FOREIGN KEY (`id_curso`) REFERENCES `curso` (`id`);
		
		ALTER TABLE `acuerdo_estudios` ADD FOREIGN KEY (`id_estudiante`) REFERENCES `estudiante` (`id_usuario`);
		
		ALTER TABLE `acuerdo_estudios` ADD FOREIGN KEY (`id_tutor`) REFERENCES `user` (`id`);
		
		ALTER TABLE `ae_asigloc_asigext` ADD FOREIGN KEY (`asig_loc`) REFERENCES `asignatura` (`id`);
		
		ALTER TABLE `ae_asigloc_asigext` ADD FOREIGN KEY (`asig_ext`) REFERENCES `asignatura_ext` (`id`);
		
		ALTER TABLE `renuncia` ADD FOREIGN KEY (`id_ae`) REFERENCES `acuerdo_estudios` (`id`);
		
		ALTER TABLE `expediente` ADD FOREIGN KEY (`id_ae`) REFERENCES `acuerdo_estudios` (`id`);
		
		ALTER TABLE `expediente` ADD FOREIGN KEY (`id_tipo_exp`) REFERENCES `tipo_expediente` (`id`);
		
		ALTER TABLE `fase_expediente` ADD FOREIGN KEY (`id_tipo_exp`) REFERENCES `tipo_expediente` (`id`);
		
		ALTER TABLE `envio_mail_fase` ADD FOREIGN KEY (`id_mail`) REFERENCES `mail_predef` (`id`);
		
		ALTER TABLE `envio_mail_fase` ADD FOREIGN KEY (`id_fase`) REFERENCES `fase_expediente` (`id`);
		
		ALTER TABLE `hist_envio_mail_fase` ADD FOREIGN KEY (`id_mail`) REFERENCES `mail_predef` (`id`);
		
		ALTER TABLE `hist_envio_mail_fase` ADD FOREIGN KEY (`id_fase`) REFERENCES `fase_expediente` (`id`);
		
		ALTER TABLE `hist_envio_mail_fase` ADD FOREIGN KEY (`id_exp`) REFERENCES `expediente` (`id`);
		
		ALTER TABLE `hist_envio_mail_fase_mod` ADD FOREIGN KEY (`id_fase`) REFERENCES `fase_expediente` (`id`);
		
		ALTER TABLE `hist_envio_mail_fase_mod` ADD FOREIGN KEY (`id_exp`) REFERENCES `expediente` (`id`);
		
		ALTER TABLE `rel_exp_fase` ADD FOREIGN KEY (`id_exp`) REFERENCES `expediente` (`id`);
		
		ALTER TABLE `rel_exp_fase` ADD FOREIGN KEY (`id_fase`) REFERENCES `fase_expedient` (`id`);
		
		ALTER TABLE `rel_exp_fase` ADD FOREIGN KEY (`id_gestor`) REFERENCES `user` (`id`);
		
		ALTER TABLE `rel_exp_fav_gestor` ADD FOREIGN KEY (`id_exp`) REFERENCES `expediente` (`id`);
		
		ALTER TABLE `rel_exp_fav_gestor` ADD FOREIGN KEY (`id_gestor`) REFERENCES `user` (`id`);
		
		ALTER TABLE `evento` ADD FOREIGN KEY (`id_creador`) REFERENCES `user` (`id`);
		
		ALTER TABLE `deadline_aviso` ADD FOREIGN KEY (`id_responsable`) REFERENCES `user` (`id`);
		
		ALTER TABLE `deadline_aviso` ADD FOREIGN KEY (`id_evento`) REFERENCES `evento` (`id`);
		
		ALTER TABLE `deadline_aviso` ADD FOREIGN KEY (`id_tarea`) REFERENCES `tarea` (`id`);
		
		ALTER TABLE `recordatorio` ADD FOREIGN KEY (`id_usuario`) REFERENCES `user` (`id`);
		
		ALTER TABLE `mensaje` ADD FOREIGN KEY (`id_emisor`) REFERENCES `user` (`id`);
		
		ALTER TABLE `mensaje` ADD FOREIGN KEY (`id_receptor`) REFERENCES `user` (`id`);
		
	\end{minted}


		
%\end{appendices}