\section*{Anexos}
\leavevmode
\begin{appendices}
	\section{Reunión de toma de contacto}
		\textit{18 de febrero de 2020}\\

		\textit{Facultad de Filosofía y Letras, Campus de Cartuja -- Universidad de Granada}\\
		
		Chelo Pérez presenta al resto del personal que trabaja en la Oficina de Relaciones Internacionales de la Facultad de Filosofía y Letras, a quienes pone a disposición para que se les haga consultas de cualquier tipo que tengan relación con el problema, pues todos conocen las dificultades a las que se han de enfrentar cada día. A continuación, Chelo presenta a Miguel Ángel Sanz, quien trabaja en la secretaría de la facultad. El interés en conocerlo se basa en que es quien ha diseñado una herramienta que resuelve parcialmente algunos de los problemas a los que se tienen que enfrentar a diario en la oficina mediante la creación de una base de datos en Access\textregistered.
		
		A pesar de su interés en la reutilización de esta herramienta que tras un año de esfuerzos está posibilitando gestionar de mejor manera la información en la oficina, se les comunica que el proyecto a construir empezaría desde cero, pudiendo así adaptarlo a otras tecnologías que facilitarían sus labores diarias considerablemente, más aún de lo que la actual solución lo hace. Es más, cabe destacar la imposibilidad de utilizar un software como es Microsoft Office\textregistered \ en un proyecto como este, dada la privatización de la licencia y las restricciones que éste pueda poner al desarrollo, que aún no se conocen, pero siempre serán menores si no se tiene fijada una herramienta sobre la cual tenga que radicar el resto del sistema.
		
		\section{Reunión de preparación del proyecto}
		
		\textit{17 de marzo de 2020}\\
		
		\textit{Google Meet}\\
		
		\textit{Participan Chelo Pérez Ocaña, Claudio López Carrascosa y Rosana Montes Soldado}\\
		
		La reunión da comienzo comentando lo hablado hasta ahora, tomando las partes esenciales del problema como punto de partida: gestión y almacenamiento de convenios, estudiantes y sus expedientes (asuntos a tratar con la secretaría). A partir de estos temas centrales, surgen aclaraciones y matices a discutir. Algunos ya se habían hablado, pero se repetían para que Rosana pudiera tomar nota y comprender el problema mejor. Otros, aunque habían sido comentados, se resalta la importancia de los mismos y se describe la forma en que se da respuesta en TWINS.
		
		Otra cosa bastante destacable es cómo Chelo describe que al principio, el usar TWINS era un verdadero desafío, pues había muchas cosas que no funcionaban y que no habían sido del todo trasladadas a la aplicación de Access, para lo que tuvo que colaborar de forma estrecha con Miguel Ángel Sanz, su creador.
		
		Sobre TWINS se comentan las cosas que se han llegado a conseguir tras los esfuerzos hechos, como una buena gestión de los convenios con otras universidades, los eventos que dispara en cuanto a envío de correos electrónicos automatizado --unos 2000 al día--. No menos importante es, la forma en que todas estas operaciones quedan registradas para su posterior volcado en un historial, donde se almacenan debidamente todos los cambios que se ha ido haciendo a cualquier información almacenada en la base de datos.
		
		En general, Chelo resalta la gran ventaja que todo ello supone, pues por ejemplo, cuando antes necesitaban dedicar alrededor de un mes al proceso de nominaciones de estudiantes, con TWINS solo precisan de una hora (y porque se va ejecutando el proceso de envío de emails independiente a cada universidad). Es más, también se resaltan mejoras como, por ejemplo, la de poner formularios a disposición de los estudiantes e integrar la información en la aplicación directamente; justo lo que han tenido que buscar para procesar la información de todos los estudiantes de movilidad una vez comenzó la pandemia por la COVID-19.
		
		Finalmente, pasamos a tratar el objetivo de mi Trabajo Fin de Grado. Con varias ideas sobre la mesa, finalmente se habla de una conversión de TWINS a la web, puesto que sería mucho más útil en cuanto a extensibilidad a otras facultades y; es más, una vez se haga el desarrollo web (con las mejoras que ya ello tiene de partida), sería más fácil integrar más servicios y añadir características. Para la realización, aparte del compromiso de Chelo para con su disponibilidad y colaboración con el trabajo, comenta la posibilidad de ceder acceso a TWINS para poder examinarlo bien y hacer un análisis completo, algo que resulta esencial para poder llevar a cabo el proyecto.
		
		\section{Reunión de presentación de TWINS}
		
		\textit{7 de abril de 2020}\\
		
		\textit{Google Meet}\\
		
		\textit{Participan Claudio López Carrascosa, Miguel Ángel Sanz Sáez y Rosana Montes Soldado}
		
		Comienza la reunión. Miguel Ángel hace uso de la palabra para hacer una pequeña presentación de TWINS. Nos explica que su nombre vino dado tras el encargo a sus hijas gemelas del diseño del logo, quienes tras explicarles el significado de los colores rojo (estudiantes salientes) y azul (estudiantes entrantes) en el ámbito de la oficina, dieron lugar a lo que es hoy por hoy el logotipo de la aplicación. Está compuesto por la palabra «gemelas» en lengua inglesa, rotulada con los dos colores. De ambos extremos de la palabra salen dos lazos que se unen en una especie de corazón morado, simbolizando la unión entre dos universidades.
		
		Acto seguido, nos comenta los objetivos de TWINS y por qué decidieron comenzar este proyecto: para manejar grandes cantidades de información, para la comunicación entre distintos actores y para lleva un correcto seguimiento de los casos de los estudiantes.
		
		También destaca aspectos de la aplicación, como el filtro de búsqueda, de suma importancia para poder localizar con eficacia a los estudiantes y/o documentos que se precisen; algo tan sencillo que ya, sin más, adelanta mucho trabajo.
		
		Terminada la presentación, comienza a hacer una demostración en directo de TWINS. Nos habla de su funcionamiento, su estructura, y nos describe todos los elementos que lo compone y que ha mencionado en la presentación.
		
		Tras aproximadamente dos horas de presentación, concluye la reunión, tras haber ido por muchos de los detalles de la aplicación (aunque no todos, por lo que probablemente se precise de más reuniones). Con todo ello, puede hacerse un análisis y así arrancar el proyecto de twinX.
		
		
\end{appendices}