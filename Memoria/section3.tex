\section{twinX}

Una vez hemos analizado el material ya existente y desde el que partiremos para la construcción de twinX, vamos a ponernos manos a la obra con su desarrollo. Si bien es cierto que no se parte desde cero en cuanto a requisitos, sí que se va a reconstruir la herramienta desde cero, para poder aportar un gran significado a todos los módulos que vayan a formar parte de twinX y así podamos cohesionarlos de una mejor manera.

\subsection{Desarrollo de twinX}

Vamos a comenzar estableciendo las ideas principales del proyecto, reafirmando el propósito y viéndolo desde otras perspectivas que, aunque parecen obvias, no siempre se tienen en cuenta. Ello nos permitirá asentar las bases del producto software final y dotarlo de calidad.

\subsubsection{Diseño centrado en el usuario}

Esta disciplina, conocida también como \textit{User-Centered Design} o UCD, destaca por basar las fases del proceso de diseño estableciendo un enfoque constante en aprender del sujeto que utilizará el producto final. Es decir, para la consecución de los objetivos, es necesario tener una retroalimentación constante por parte del usuario final, que será quien vaya orientando nuestros avances según sea su interacción con lo que vayamos desarrollando.

El proceso dispone de unas fases a seguir, como son:
\begin{itemize}
	\item \textbf{Especificación del contexto de uso:} quiénes usarán el producto, para qué y bajo qué condiciones lo harán.
	\item \textbf{Especificación de requisitos:} identificar los objetivos que tienen que cumplirse para dejar a los usuarios satisfechos.
	\item \textbf{Crear soluciones de diseño:} en distintas etapas, desde un concepto poco definido hasta un diseño completo.
	\item \textbf{Evaluación de los diseños:} a través de las pruebas con los usuarios, de forma ideal.
\end{itemize}

\paragraph{Pautas de Accesibilidad}
\leavevmode\\[\baselineskip]

El no atender a causas de accesibilidad sería no aplicar correctamente, de alguna forma, el tipo de diseño escogido para el desarrollo del producto. Cuando se hacen pruebas, el objetivo no es otro que adaptar el producto para que pueda ser bien utilizado por el mayor número de usuarios posible y de la forma satisfactoria para ellos. Es por ello por lo que tenemos que abrir el abanico y contemplar que pueden ser numerosos los usuarios que necesiten adaptaciones para interaccionar correctamente con el contenido web.

Para ello, se propone la utilización de los recursos especificados en \textit{Web Content Accessibility Guidelines} (WCAG) \citep{wcag}.

%%%%%% SECCIÓN 3.5
%\subsection{Descripción de la propuesta}
%
%Para desarrollar twinX primero se ha de elegir una metodología. En este caso, dado que este proyecto es tan sólo un comienzo para lo que sería el producto final y que tiene unos requisitos poco cerrados, escogeremos una metodología de desarrollo ágil.
%
%Este tipo de metodologías se basan en la adaptación a medios con necesidades cambiantes, con un desarrollo muy de cerca con el cliente, quien está presente junto con el equipo de desarrollo en las numerosas reuniones que se hacen a lo largo de la vida del proyecto. En este caso, el equipo de desarrollo es ajustado, pero no por ello dejaremos de lado la necesaria intervención de dos personas que hacen el papel de clientes y que son:
%
%\begin{itemize}
%	\item María Consuelo Pérez Ocaña (Coordinadora de Internacionalización en la la Facultad de Filosofía y Letras)
%	\item Miguel Ángel Pérez Sanz (personal de secretaría y creador de TWINS)
%\end{itemize}