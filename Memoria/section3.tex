\section{twinX}
\subsection{Descripción de la propuesta}

Para desarrollar twinX primero se ha de elegir una metodología. En este caso, dado que este proyecto es tan sólo un comienzo para lo que sería el producto final y que tiene unos requisitos poco cerrados, escogeremos una metodología de desarrollo ágil.

Este tipo de metodologías se basan en la adaptación a medios con necesidades cambiantes, con un desarrollo muy de cerca con el cliente, quien está presente junto con el equipo de desarrollo en las numerosas reuniones que se hacen a lo largo de la vida del proyecto. En este caso, el equipo de desarrollo es ajustado, pero no por ello dejaremos de lado la necesaria intervención de dos personas que hacen el papel de clientes y que son:

\begin{itemize}
	\item María Consuelo Pérez Ocaña (Coordinadora de Internacionalización en la la Facultad de Filosofía y Letras)
	\item Miguel Ángel Pérez Sanz (personal de secretaría y creador de TWINS)
\end{itemize}