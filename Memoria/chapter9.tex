\chapter{Conclusiones}

Hemos analizado una herramienta creada para llevar a cabo el trabajo de la oficina de relaciones internacionales de la Universidad de Granada: sus carencias, sus puntos fuertes y por qué es necesaria. A partir de ello, hemos definido y diseñado las pautas para la consecución de una mejora integral de la misma, con grandes posibilidades de extensión y continuación en el futuro de su desarrollo para automatizar y mejorar aún más el trabajo de los gestores. Como resultado del posterior desarrollo, se ha obtenido una plataforma piloto capaz de resolver la gran mayoría de los problemas diarios en la oficina de la que partir para seguir cubriendo todas las necesidades actuales y ser ejemplo de gestión de la internacionalización en toda la institución.

A continuación, vamos a hacer un sumario sobre cómo ha sido el curso del proyecto en general.

\section{Consecución de los objetivos propuestos}

\subsection{Objetivos generales}

Ya en la sección introductoria \ref{sec:objetivosgen} se expusieron cuáles eran las claras intenciones de twinX. A grandes rasgos, sí se ha demostrado la capacidad de poder \textbf{prescindir de Access\textregistered} como herramienta principal de trabajo; aunque no de inmediato, pero sí en un futuro donde se continuara con el desarrollo de twinX.

Sobre las mejoras que presenta twinX, generalmente a nivel visual, y a nivel interno, las tablas en base de datos se han simplificado grotescamente, dotando la solución de una eficiencia mayor a la de TWINS. El rediseño ha supuesto el cuestionarse la necesidad de muchos elementos que en TWINS simplemente, están. Se han ido haciendo arreglos y se ha refactorizado muy poco código o directamente no se ha hecho, lo que lleva a tener una herramienta que externamente es funcional y aporta soluciones, pero que internamente carga en su mochila decisiones críticas para su estabilidad. Pensemos que el hecho de que twinX tenga los datos guardados en un servidor, donde todos los usuarios tengan la misma versión de la información es un avance muy grande, y por tanto es necesario plantearse la necesidad de implementar un sistema como twinX.

Acerca de las entidades genéricas que TWINS maneja: convenios, estudiantes, expedientes y un largo etcétera, podemos decir que se han portado con éxito a twinX, tras la creación del modelo de la base de datos, con 30 tablas que, aunque sean pocas en comparación con las que alberga TWINS, ya pueden resolver muchos problemas sin necesidad de tantos modelos. 

Sin embargo, aunque ha formado parte de la planificación, la intervención de estudiantes (salientes y entrantes) y tutores no se ha terminado llevando a cabo, dada la envergadura del proyecto. Aún así, se ha tenido en cuenta en la elaboración del modelo de la base de datos (figura  \ref{fig:modeloBD}) donde, por ejemplo, existen estados para los acuerdos de estudios (dependiendo de si el tutor aprueba o no el acuerdo), relaciones de asignaturas locales y externas con acuerdos, y la contemplación de estudiantes y tutores como usuarios de la plataforma. Esto vería la luz en los sprints 3 y 4 \ref{subsec:sprints} que no se han planificado por motivos de tiempo.

\subsection{Objetivos específicos}

En la sección \ref{subsec:propuesta} se propusieron unas pautas específicas que definían las capacidades de twinX. Sobre ello, cabe destacar el distinto enfoque con que se han tratado los convenios. Si bien en TWINS existían dos vistas, en twinX tenemos una sola vista. El resultado ha sido intencionado, y es que se tomó la decisión de dividir la información del convenio en distintas partes con desplegables para visualizarla de forma diseminada, y no concentrada en un único punto. La idea surgió dada la segregación de información en los convenios de TWINS, y salvo por alguna actitud de los usuarios que no se puede atribuir directamente al mal diseño de los nuevos convenios, es considerado un acierto por simplificar la parte de convenios, pues ya tiene suficiente contenido que la densifica.

Respecto al módulo de calendario y la implementación del calendario y las tareas, tal y como comentamos al término de la sección \ref{implementacion}, no ha sido finalmente llevada a cabo. El motivo no es otro que el priorizar otras mejoras y respaldo de las funciones ya creadas para el poco tiempo restante del que se disponía. En su lugar, se han integrado los recordatorios, que dotan de sentido la existencia del módulo de calendario y que sustituye de una manera el vacío que deja el no poder haber llevado a cabo las dos historias de usuario \hyperref[tab:HU3.1]{HU3.1} y \hyperref[tab:HU3.2]{HU3.2}.

Por lo demás, la implementación ha cubierto en su gran mayoría los objetivos especificados en la propuesta del producto, por lo que podemos afirmar que su llevada a cabo ha sido satisfactoria.

\section{Valoración del grado de viabilidad de la continuación del proyecto}

Lo más importante a la hora de considerar el éxito de twinX es remitirse a la opinión de los usuarios finales. Para ello han sido esenciales \hyperref[pruebas]{las pruebas}, donde hemos podido confirmar que la experiencia tras probar twinX ha sido muy satisfactoria. Los usuarios con los que se ejecutaron los tests de usabilidad admiraron mucho los avances que se habían hecho con el proyecto y agradecieron el esfuerzo realizado.

En cuanto a implementación de los dos últimos sprints planificados con el proyecto, tan solo se necesitaría más tiempo y una oferta de continuación con el proyecto bajo el amparo de la ORI-FyL, donde la disciplina de diseño escogida (\textit{User-centered design}) tomaría mucho mayor potencial que en la realización de twinX, donde ya se tenía un material sobre el que trabajar y un profundo conocimiento sobre las necesidades tras las reuniones con los usuarios y el haber vivido la experiencia de realización de un programa de movilidad en primera persona. Con ello, al menos los primeros meses, no se avanzaría demasiado en cuanto a inclusión de nuevas características (o al menos no con un elevado ritmo), dado que habría muchos detalles que pulir y que, aunque la aproximación que oferta twinX no deje de ser buena, no está hecha a prueba de fuego con el trabajo diario del personal de secretaría.

También sería importante el considerar la adquisición de nuevos conocimientos a varios niveles para poder consolidar el proyecto, de modo que no llegue a alcanzar etapas muy avanzadas pero sin tener una base consolidada; esto es, una robusta y confiable aplicación que siga resolviendo problemas. Habría incluso que sumar más gastos de infraestructura, como son los derivados de la contratación de una mayor capacidad de procesamiento y/o almacenaje para hospedar twinX en un servidor, sobre todo teniendo en cuenta que se espera que numerosos estudiantes de todo el mundo se conecten a la plataforma.

Sin embargo, y a pesar de los contras, sería una apuesta segura, ya que es un proyecto con mucho futuro por delante y que resuelve un gran problema como es el arrastrar con los errores y dificultades de TWINS, para el cual podríamos encontrar posteriormente la necesidad de reconstruir un sistema que, para entonces, tendrá poca (o no tendrá) tolerancia a fallos y que será mucho más costoso rehacer en cuanto a tiempo y a dinero.

\section{Trabajos futuros}
\label{sec:trabajosfuturos}

Independientemente de cuál sea el curso de twinX en el futuro, podemos identificar una serie de necesidades provenientes de la observación del estado de la aplicación y de las conclusiones obtenidas tras las sesiones de pruebas:

\begin{itemize}
	\item \textbf{Refactorización del código:} nunca viene mal echar un vistazo a las primeras líneas programadas en una tecnología totalmente nueva. Con ello, se mejoraría bastante la eficiencia y se ahorraría mucho código. Por ejemplo, la vista compuesta del envío de mails predefinidos (en fases de expedientes) no tuvo exactamente la misma implementación que la de las fases de un expediente. Siendo ésta última la más reciente, es obvio que su código es de mayor calidad y tiene algunos archivos de vista menos que la primera aproximación propuesta.
	\item \textbf{Creación de nuevos accesos directos:} por ejemplo, para ir desde el estudiante a sus expedientes no tenemos ningún atajo que nos libre de pasar primero por el menú de expedientes y la correspondiente búsqueda del estudiante en cuestión.
	\item \textbf{Creación de un bot de Telegram:} para recibir avisos de mensajes o recordatorios en el caso del gestor o, en futuros sprints, de revisión del acuerdo de estudios para el estudiante, sin necesidad de tener que comprobar constantemente la plataforma.
	\item \textbf{Histórico del convenio a modo de versiones:} creación de una tabla que tan solo difiera en un atributo más que almacenaría la fecha y hora de la versión, para que siempre se puedan recuperar versiones antiguas de un convenio, en caso que fuera necesario.
	\item \textbf{Histórico completo de un estudiante:} ver de un vistazo, a modo de línea de tiempo, los cambios que se han hecho en su acuerdo, los expedientes que se han abierto o cerrado con sus fases, incluso sus mensajes, etc.
	\item \textbf{Gestión de las comunicaciones con los estudiantes mediante un sistema de tiques:} asociando un número identificativo a un caso y al que puedan etiquetar, para poder ser filtrado posteriormente por los gestores.
\end{itemize}

\section{Lecciones aprendidas}

Con la creación de twinX ha habido un constante proceso de aprendizaje desde el primer minuto. Ya en febrero, cuando se celebró la primera reunión, comencé la lectura de un libro de PHP, a sabiendas del Framework a usar, a través de la sugerencia de la tutora académica. Es cierto que cursar la asignatura \textit{Sistemas de Información Basados en Web} ayudó bastante a impulsar el aprendizaje, pero luego vino la realización del extenso tutorial de Yii y resultó ser como una vuelta a empezar de cero. Si bien ya comprendía muchas conjeturas del lenguaje, el framework resultó ser algo totalmente nuevo y extraño. Otros como React o Angular están basados en JavaScript y no es muy complicado aprender a manejarlos; pero esto no es debido al lenguaje, sino al potencial. Yii permite ahorrar mucho tiempo con su motor generador de código, algo que no todos los demás tienen. Y es que son tantas las características que posee y lo mucho que lo simplifica todo que obligan al desarrollador a comprender a la perfección para qué se usan determinadas clases y cuáles son las pautas de uso de sus herramientas. Esto es, en parte, bueno, ya que al «forzar» al usuario con las prácticas propias se evita la escritura de código mal estructurado o que funciona, pero que simplemente no usa bien el framework.

Además, también he hecho mi primer despliegue, por lo que AWS ha sido algo nuevo de igual forma. Por suerte, no ha sido muy complicado configurar el servidor y, de haberlo sido, hay suficiente documentación en la web, por lo que no debería haber habido problema por esa parte.

Para la redacción de esta memoria, a pesar de haber partido con conocimiento sobre código en \LaTeX, nunca había redactado algo tan extenso y completo como este documento, por lo que también ha sido necesario emplear varias horas para afianzar conocimientos y aprender otros nuevos.

En la base, para poder catapultarme hasta la finalización de este proyecto se encuentran asignaturas como \textit{Fundamentos de Bases de Datos}, \textit{Fundamentos y Metodología de la Programación}, \textit{Metodologías de Desarrollo Ágil}, \textit{Programación y Diseño Orientado a objetos} y otras extranjeras como \textit{User Experience and Design of User Interfaces and Services} y \textit{User Interface Programming} durante mi estancia en República Checa como estudiante Erasmus.

Al margen del aprendizaje técnico, en mi posición, me veo habiendo diseñado un sistema de información completo y funcional que soluciona un problema y cuyo resultado ha sido muy positivo, según el juicio de los usuarios finales. Estoy muy agradecido de haber realizado mi trabajo fin de grado en el ámbito de la ingeniería del software, que en el fondo no pretende otra cosa que ayudar a los demás a solucionar sus problemas y a hacer las cosas de una forma más apetecible y actual.