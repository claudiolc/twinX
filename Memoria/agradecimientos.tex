\chapter*{Agradecimientos}

Hoy termino lo que comencé hace aproximadamente cuatro años. Siempre tuve más o menos claro que la informática era lo mío. Todo comenzó cuando mi padre me inició en el mundo de los ordenadores montándome mi primer equipo cuando tenía cerca de seis años. Mi madre nos oía extrañada --y lo sigue haciendo-- cuando hablábamos de asuntos que se salían de su entendimiento. Es cierto que ninguno de ellos me ha enseñado conceptos muy avanzados sobre informática, pero cuando uno llega al fin de una etapa como esta se da cuenta de que hay muchas más cosas como la actitud, el respeto y el ser buena persona que influyen como lo que más y son las herramientas más valiosas. Y en eso han sido mis principales maestros, pues si hoy soy quien soy, no es más que gracias a ellos y a su constante esfuerzo por que mejore como persona día a día. 

Ellos me hicieron el regalo de traer a mi hermana al mundo, quien también me ha enseñado lecciones tan importantes como las de mis padres, y a la que quiero con toda mi alma por endulzar mi vida y hacerme crecer aún más como persona. Somos muy distintos en lo externo, pero en lo interno ambos sabemos que somos como dos gotas de agua.

No, la familia no se escoge, pero si tuviera que hacerlo os volvería a escoger a vosotros mil veces más. Gracias, gracias y gracias.

Junto a mí estos cuatro años también han estado mis amigas Beatriz y Mercedes. Por acompañarme, me hicieron hasta una visita durante mi estancia en el extranjero, y es que no son otra cosa que mis compañeras de viaje en la vida. Me han apoyado como nadie y han estado ahí en las tardes de estudio, en los momentos más buenos y en los no tan buenos también. No concibo la vida sin ellas, y por ello les debo mucho. Gracias, hermanas.

Al llegar a Granada me encontré con quien sin saberlo se ha convertido en otro pilar importante en mi vida académica y ya cotidiana: mi compañero de piso Diego García. Unos años después, la ETSIIT me descubrió a Diego Cortés. Hoy día sigo compartiendo hogar con ellos, quienes también han estado al pie del cañón para escucharme cuando lo necesitaba y para compartir vivencias juntos. Os quiero.

A mis abuelos, que siempre me preguntan por los estudios y siempre velan por mi seguridad y bienestar.

A mis amigos de clase, con los que tampoco habría sido nada igual: Ceci, Carlos, Nabil, Rafa, Pablo, Dani, Enrique. Gracias por hacer mis tardes en la escuela mucho más amenas, por darme tantos buenos momentos y por haberme acogido con vosotros tan deprisa. Sois de lo mejor que me llevo.

A mi tutora Rosana, por confiar en mí para la realización de este proyecto y por apoyarme para sacar adelante mi propio tema cuando tenía dudas sobre su viabilidad. Sin tu empujón probablemente no lo hubiera disfrutado tanto como lo he hecho.

A Chelo y Miguel Ángel por estar ahí para atenderme siempre que he necesitado y por dedicarme su tiempo. Sin vosotros, este proyecto no se podría haber llevado a cabo.

Incluso a todos aquellos que me reconocen como «el delegado», mis compañeros de promoción de segundo. Gracias por vuestra admiración, el reconocimiento y la gratitud es siempre lo que me mantiene en pie.

A profesores como Olga Pons, Jesús García, Segio Alonso, Carlos Cano, Ana Anaya: de ellos, aparte de los conceptos, me llevo vuestro entusiasmo por la docencia. Espero que en el resto de sus carreras se encuentren con muchos más estudiantes con sed de aprendizaje. Les deseo lo mejor.

A demás familiares, amigos y allegados que han confiado en mí y en mi valía.

Y a ti, lector, por tu interés en leer mi trabajo.

A todos vosotros,

Gracias.

De corazón.


