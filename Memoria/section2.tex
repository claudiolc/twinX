\section{Caso de estudio: Oficina de Relaciones Internacionales de la Facultad de Filosofía y Letras de la Universidad de Granada}
\subsection{Sobre la oficina}
La labor principal de la oficina es encargarse de la gestión de los programas de intercambio y la movilidad de estudiantes, incluso de programas de intercambio de profesores. Como no podía ser de otra manera, proporcionan la información necesaria para los estudiantes interesados en estos programas, tanto fuera de la UGR como para acoger a estudiantes extranjeros.

En su día a día, atienden peticiones y dudas de los estudiantes que participan en alguno de los citados programas; es más, se dedican a asesorar y mostrar todas las alternativas de las que disponen cuando se les presenta alguna situación complicada, de modo que puedan resolverlo a de la forma en que más les beneficie. Trabajan, en definitiva, con el futuro de los estudiantes, pues la movilidad la desarrollan con el objetivo de complementar su formación, algo fundamental y abrumador al mismo tiempo cuando se cruzan fronteras y se quiere seguir en el camino de la educación en una universidad que no es la de casa.

La oficina se sitúa junto a la secretaría, en la Facultad de Filosofía y Letras de la Universidad de Granada, en el Campus de Cartuja, y en ella trabajan cinco personas. 