\section{Caso de estudio: Oficina de Relaciones Internacionales de la Facultad de Filosofía y Letras de la Universidad de Granada}
\subsection{Sobre la oficina}
La labor principal de la oficina es encargarse de la gestión de los programas de intercambio y la movilidad de los estudiantes e incluso de profesores. Como no podía ser de otra manera, proporcionan la información necesaria para los interesados en estos programas, tanto fuera de la UGR como para acoger a estudiantes extranjeros. Es más, también revisan convenios existentes con las universidades y hacen otros nuevos para ofrecer cada vez más alternativas para poder mejorar nuestra formación.

En su día a día, atienden peticiones y dudas de los estudiantes que participan en alguno de los citados programas; es más, se dedican a asesorar y mostrar todas las alternativas de las que disponen cuando se nos presenta alguna situación complicada, de modo que podamos resolverlo de la forma en que más nos beneficie. Trabajan, en definitiva, con el futuro de los estudiantes, pues la movilidad la desarrollamos con el objetivo de complementar nuestra formación, algo fundamental y abrumador al mismo tiempo cuando se cruzan fronteras y se quiere seguir en el camino de la educación en una universidad que no es la de casa.

La oficina se sitúa junto a la secretaría, en la Facultad de Filosofía y Letras de la Universidad de Granada, en el Campus de Cartuja, y en ella trabajan alrededor de cuatro personas. Es, dadas las cifras que se tienen, una gran cantidad de información las que tan sólo unas pocas personas tienen que manejar con una herramienta que ha sido creada sobre la marcha para facilitar su importante labor; un trabajo que no puede parar, pues los estudiantes de movilidad son uno de los pilares fundamentales de la universidad.

\subsection{Servicio al estudiantado}
\subsubsection{Estudiantes salientes o \textit{outcoming}}

En relación con los estudiantes salientes, en la oficina se encargan de coordinar a los tutores académicos, que son quienes revisan los acuerdos de estudios que los estudiantes proponen para iniciar su movilidad. Una vez éstos les han dado el visto bueno, en la oficina revisan cada uno de los mismos para asegurarse de que todo está en orden. Una vez iniciada la movilidad puede darse el caso en que el/la estudiante desee hacer alguna modificación a su acuerdo, debido a algún cambio imprevisto o a que alguna asignatura no fuera como se esperaba en el destino. En ese caso, el proceso sería el mismo.

También escuchan casos de estudiantes con problemas particulares y que deben ser examinados con detenimiento, para ofrecer la mejor alternativa, ya sea hablando con los coordinadores en los destinos o arreglando algún dato en los convenios que haya causado algún inconveniente en la movilidad de algún(a) estudiante. De este modo, las futuras movilidades podrán hacerse con una mayor posibilidad de éxito, sobre todo cuando se trate de convenios nuevos. Son múltiples los casos en que se deba necesitar asistencia: una lengua de impartición de las asignaturas distinta a la esperado, un nivel requerido en un idioma que no se había comunicado al estudiante, etc.

Una vez el/la estudiante vuelve de su movilidad, se inicia el proceso de reconocimiento de créditos, para el cual se establecen unas correspondencias entre las calificaciones obtenidas en el destino y las que se van a especificar en su expediente en la UGR. Para ello, esta información debe estar reflejada en un documento oficial y que la secretaría pueda aceptar, por lo que en muchos casos el personal tiene que ponerse en contacto con los responsables en el destino y solicitar los certificados que sean precisos. De esta manera, se puede tener la certeza de que es alguien de confianza quien los remite, ya que se ha tomar muy en serio la veracidad e los mismos.

\subsubsection{Estudiantes entrantes o \textit{incoming}}