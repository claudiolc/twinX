\chapter{Diseño de la interfaz de usuario}

Una vez hemos definido cuáles son nuestros objetivos, analizado el estado del arte y establecido unas pautas para el desarrollo, vamos a comenzar con el diseño de la parte visual. Esto es, a su vez, un proceso que requiere de distintas definiciones, de manera que al término de este capítulo, no sólo tengamos el diseño de las interfaces de usuario, sino que también comprendamos por qué se han diseñado así y la intención de cada elemento que disponemos en pantalla.

\section{Matriz de tareas de usuario}

Un buen comienzo sería el de definir los roles de usuario. Hemos hablado ya en el capítulo anterior sobre la figura del \gls{gestortwinX} y la del \gls{superusuario}. Incluyamos entonces también las de \gls{Tutor}, estudiante y, de forma excepcional, la de coordinador externo. Sobre este último, recordemos que no tenía ningún tipo de interacción en TWINS y que hasta entonces, los usuarios de la plataforma habían establecido intercambios de información por correo electrónico con los coordinadores de otras facultades, que es el método estándar de realizar las \glspl{Nominacion}. Sin embargo, aunque no para sus primeras versiones, se espera que twinX brinde la capacidad de hacer estas nominaciones a través de una interfaz gráfica, de una forma mucho más sencilla que con el envío de un texto carente de una respuesta automatizada o que dependa de otro actor externo. Pensemos que el intercambio de información con ese método puede hacerse de cualquier forma, pues el cuerpo de un email no da indicaciones acerca de cómo se tiene que presentar esta información: es simplemente un campo de introducción de texto.

[continuación]

\section{Sitemap}
\section{Labelling (iconografía)}
\section{Bocetos wireframe}

\subsection{Wireframes del módulo de gestión}

\begin{figure}
	\centering
	\includegraphics[width=\textwidth]{img/Wireframes/Gestión/dashboard.png}
	\caption[Wireframe de \textit{Dashboard}]{Wireframe del \textit{Dashboard} donde el usuario verá de un vistazo toda la información de mayor importancia al iniciar la aplicación}
	\label{fig:dashboardWF}
\end{figure}

\begin{figure}
	\centering
	\includegraphics[width=\textwidth]{img/Wireframes/Gestión/convenios_lista.png}
	\caption{Wireframe de lista de convenios}
	\label{fig:convenios_listaWF}
\end{figure}

\begin{figure}
\centering
\includegraphics[width=\textwidth]{img/Wireframes/Gestión/vista_convenio_básica_contraída.png}
\caption[Wireframe de vista de convenio básica]{Wireframe de vista de convenio básica. Secciones contraídas.}
\label{fig:vista_conv_básica_contWF}
\end{figure}

\begin{figure}
	\centering
	\includegraphics[width=\textwidth]{img/Wireframes/Gestión/vista_convenio_avanzada_contraída.png}
	\caption[Wireframe de vista de convenio avanzada]{Wireframe de vista de convenio avanzada. Secciones contraídas. Acceso desde el botón «+» de la botonera en la esquina superior derecha del contenido principal.}
	\label{fig:vista_conv_avanzada_contWF}
\end{figure}

\begin{figure}
	\centering
	\includegraphics[width=\textwidth]{img/Wireframes/Gestión/expedientes_lista.png}
	\caption{Wireframe de lista de expedientes}
	\label{fig:expedientes_listaWF}
\end{figure}

\begin{figure}
	\centering
	\includegraphics[width=\textwidth]{img/Wireframes/Gestión/estudiantes_lista.png}
	\caption{Wireframe de lista de estudiantes}
	\label{fig:estudiantes_listaWF}
\end{figure}

\begin{figure}
	\centering
	\includegraphics[width=\textwidth]{img/Wireframes/Gestión/búsqueda.png}
	\caption[Wireframe de búsqueda]{Wireframe de búsqueda. Diálogo modal, desplegable desde la barra. Capacidad de filtrado}
	\label{fig:búsquedaWF}
\end{figure}

\begin{figure}
	\centering
	\includegraphics[width=\textwidth]{img/Wireframes/Gestión/expediente_detalle.png}
	\caption{Wireframe de detalle de expediente}
	\label{fig:expediente_detalleWF}
\end{figure}

\begin{figure}
	\centering
	\includegraphics[width=\textwidth]{img/Wireframes/Gestión/cambio_fase.png}
	\caption[Wireframe del modal de cambio de fase]{Wireframe del modal de cambio de fase. Advierte sobre las acciones a llevar a cabo tras la tramitación del cambio de fase en un expediente concreto.}
	\label{fig:cambio_faseWF}
\end{figure}

\begin{figure}
	\centering
	\includegraphics[width=\textwidth]{img/Wireframes/Gestión/estudiante_detalle.png}
	\caption{Wireframe de la vista de detalle de un estudiante}
	\label{fig:estudiante_detalleWF}
\end{figure}

%%PROVISIONAL PARA SEPARAR AMBOS CONJUNTOS DE WIREFRAMES
\newpage

\subsection{Wireframes del módulo de calendario}

\begin{figure}
	\centering
	\includegraphics[width=\textwidth]{img/Wireframes/Calendario/eventos_lista.png}
	\caption{Wireframe de lista de eventos}
	\label{fig:eventos_listaWF}
\end{figure}

\begin{figure}
	\centering
	\includegraphics[width=\textwidth]{img/Wireframes/Calendario/notificaciones_lista.png}
	\caption[Wireframe de lista de notificaciones]{Wireframe de lista de notificaciones. Pueden aparecer tres tipos de notificaciones, dependiendo del tipo de mensaje recibido o generado por la propia plataforma (a través de avisos y/o recordatorios)}
	\label{fig:notificaciones_listaWF}
\end{figure}

\begin{figure}
	\centering
	\includegraphics[width=\textwidth]{img/Wireframes/Calendario/nuevo_evento.png}
	\caption[Wireframe de creación de un evento]{Wireframe de creación de un evento. Se distinguen dos vistas: la información del evento en general (izquierda) y la de sus subtareas (derecha)}
	\label{fig:nuevo_eventoWF}
\end{figure}



